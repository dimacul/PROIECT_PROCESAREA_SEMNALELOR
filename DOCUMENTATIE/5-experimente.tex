\chapter{Experimente și Rezultate}

Acest capitol prezintă evaluarea experimentală a implementării algoritmului Gorilla. Sunt descrise seturile de date utilizate, metodologia de testare și rezultatele obținute.

\section{Seturi de date}

Pentru validarea implementării au fost utilizate două seturi de date cu caracteristici diferite:

\subsection{CPU Load Average (serie univariată)}

Primul set de date conține metrici de încărcare CPU colectate de pe un sistem Linux.

\begin{table}[h]
\centering
\begin{tabular}{@{}ll@{}}
\toprule
\textbf{Caracteristică} & \textbf{Valoare} \\
\midrule
Număr de puncte & 480 \\
Perioadă de colectare & $\sim$16 ore \\
Interval mediu & $\sim$120 secunde \\
Tip date & Univariat (load average) \\
Interval valori & 0.57 -- 2.55 \\
\bottomrule
\end{tabular}
\caption{Caracteristicile setului de date CPU Load}
\label{tab:cpu-dataset}
\end{table}

\textbf{Load average} reprezintă numărul mediu de procese în coadă de execuție pe o fereastră de timp. Valorile variază ușor în timp, cu schimbări graduale caracteristice sarcinilor tipice de server.

\subsection{Room Climate Dataset (serie multivariată)}

Al doilea set de date provine din Room Climate Dataset, o colecție de măsurători de la senzori de mediu interior \cite{roomclimate}.

\begin{table}[h]
\centering
\begin{tabular}{@{}ll@{}}
\toprule
\textbf{Caracteristică} & \textbf{Valoare} \\
\midrule
Număr de puncte & 68,229 \\
Perioadă de colectare & $\sim$22 zile \\
Interval mediu & $\sim$28 secunde \\
Număr variabile & 8 \\
Node ID filtrat & NID = 1 \\
\bottomrule
\end{tabular}
\caption{Caracteristicile setului de date Room Climate}
\label{tab:room-dataset}
\end{table}

Cele 8 variabile măsurate sunt:

\begin{enumerate}
    \item \textbf{Temperatură} (°C): 17--26°C
    \item \textbf{Umiditate relativă} (\%): 25--60\%
    \item \textbf{Lumină 1} (lux): senzor principal de lumină
    \item \textbf{Lumină 2} (lux): senzor secundar
    \item \textbf{Ocupare}: indicator binar (0/1)
    \item \textbf{Activitate}: nivel de mișcare detectată
    \item \textbf{Ușă}: stare ușă (deschis/închis)
    \item \textbf{Fereastră}: stare fereastră (deschis/închis)
\end{enumerate}

\section{Metodologia experimentală}

\subsection{Metrici de evaluare}

Au fost calculate următoarele metrici:

\begin{enumerate}
    \item \textbf{Dimensiune originală}: $n \times (8 + 8k)$ bytes, unde $n$ este numărul de puncte și $k$ numărul de variabile (1 pentru univariat).

    \item \textbf{Dimensiune comprimată}: numărul de bytes după compresie.

    \item \textbf{Rata de compresie}:
    \begin{equation}
        R = \frac{\text{Dimensiune originală}}{\text{Dimensiune comprimată}}
    \end{equation}

    \item \textbf{Bytes per punct}:
    \begin{equation}
        B = \frac{\text{Dimensiune comprimată}}{n}
    \end{equation}

    \item \textbf{Economie procentuală}:
    \begin{equation}
        E = \left(1 - \frac{1}{R}\right) \times 100\%
    \end{equation}
\end{enumerate}

\subsection{Configurația blocurilor}

Testele au fost realizate cu blocuri de 2 ore (7.200.000 ms), conform configurației standard Gorilla.

\subsection{Validare round-trip}

Pentru a verifica corectitudinea implementării, fiecare test include o validare \textit{round-trip}: datele sunt comprimate, apoi decomprimate, iar valorile rezultate sunt comparate bit-cu-bit cu originalele.

\section{Rezultate: Serie univariată (CPU Load)}

\subsection{Statistici de compresie}

\begin{table}[h]
\centering
\begin{tabular}{@{}lr@{}}
\toprule
\textbf{Metrică} & \textbf{Valoare} \\
\midrule
Puncte totale & 480 \\
Blocuri (2 ore) & 5 \\
Dimensiune originală & 7,680 bytes \\
Dimensiune comprimată & 3,412 bytes \\
Rata de compresie & 2.25x \\
Bytes per punct & 7.11 \\
Economie & 55.6\% \\
\bottomrule
\end{tabular}
\caption{Rezultate compresie pentru CPU Load}
\label{tab:cpu-results}
\end{table}

\subsection{Analiza rezultatelor}

Rata de compresie de 2.25x este mai modestă decât cea raportată în articolul Gorilla ($\sim$12x) din următoarele motive:

\begin{enumerate}
    \item \textbf{Interval neregulat}: Timestamp-urile din acest set nu sunt perfect periodice (interval mediu de $\sim$120s cu variații), ceea ce reduce eficiența delta-of-delta.

    \item \textbf{Valori variabile}: Load average variază frecvent, ceea ce produce mai multe XOR-uri non-zero.

    \item \textbf{Set de date mic}: Cu doar 480 de puncte, overhead-ul primelor valori (scrise integral) are impact proporțional mai mare.
\end{enumerate}

\section{Rezultate: Serie multivariată (Room Climate)}

\subsection{Statistici de compresie}

\begin{table}[h]
\centering
\begin{tabular}{@{}lr@{}}
\toprule
\textbf{Metrică} & \textbf{Valoare} \\
\midrule
Puncte totale & 68,229 \\
Variabile & 8 \\
Blocuri (2 ore) & 262 \\
Dimensiune originală & 4,912,488 bytes \\
Dimensiune comprimată & 1,190,271 bytes \\
Rata de compresie & 4.13x \\
Bytes per punct & 17.45 \\
Economie & 75.8\% \\
\bottomrule
\end{tabular}
\caption{Rezultate compresie pentru Room Climate}
\label{tab:room-results}
\end{table}

\subsection{Analiza pe variabile}

Diferitele variabile din setul Room Climate au comportamente distincte care afectează compresia:

\begin{itemize}
    \item \textbf{Temperatură și umiditate}: Variază lent și continuu, producând multe XOR-uri cu puțini biți semnificativi. Compresie foarte bună.

    \item \textbf{Variabile binare} (ocupare, ușă, fereastră): Valorile sunt 0 sau 1, producând fie XOR = 0 (identic), fie diferențe mari când se schimbă starea.

    \item \textbf{Senzori de lumină}: Variabilitate moderată, dar cu tendința de a rămâne constanți pe perioade lungi (noapte vs. zi).
\end{itemize}

\subsection{Comparație cu stocarea separată}

Dacă am fi tratat fiecare variabilă ca o serie separată (duplicând timestamp-urile), dimensiunea ar fi fost:

\begin{equation}
    \text{Dimensiune separată} = 8 \times 68,229 \times 16 = 8,733,312 \text{ bytes}
\end{equation}

Stocarea multivariată (un singur stream de timestamp-uri) economisește:
\begin{equation}
    8,733,312 - 4,912,488 = 3,820,824 \text{ bytes} \approx 3.6 \text{ MB}
\end{equation}

doar din evitarea duplicării timestamp-urilor, înainte de compresie.

\section{Distribuția biților}

\subsection{Timestamp-uri}

Analiza distribuției valorilor delta-of-delta pentru setul Room Climate:

\begin{table}[h]
\centering
\begin{tabular}{@{}lrr@{}}
\toprule
\textbf{Bucket} & \textbf{Procent} & \textbf{Biți} \\
\midrule
$D = 0$ & $\sim$85\% & 1 \\
$D \in [-64, 63]$ & $\sim$12\% & 9 \\
$D \in [-256, 255]$ & $\sim$2\% & 12 \\
$D \in [-2048, 2047]$ & $\sim$0.8\% & 16 \\
Altfel & $\sim$0.2\% & 36 \\
\bottomrule
\end{tabular}
\caption{Distribuția delta-of-delta pentru Room Climate}
\label{tab:dod-dist}
\end{table}

Media ponderată: $\sim$2.3 biți per timestamp.

\subsection{Valori}

Distribuția tipurilor de codare XOR:

\begin{table}[h]
\centering
\begin{tabular}{@{}lrr@{}}
\toprule
\textbf{Caz} & \textbf{Procent} & \textbf{Biți medii} \\
\midrule
XOR = 0 (identic) & $\sim$45\% & 1 \\
Refolosire fereastră & $\sim$35\% & $\sim$25 \\
Fereastră nouă & $\sim$20\% & $\sim$40 \\
\bottomrule
\end{tabular}
\caption{Distribuția codării XOR pentru Room Climate}
\label{tab:xor-dist}
\end{table}

\section{Efectul dimensiunii blocului}

S-a testat impactul dimensiunii blocului asupra ratei de compresie:

\begin{table}[h]
\centering
\begin{tabular}{@{}lcc@{}}
\toprule
\textbf{Dimensiune bloc} & \textbf{Rata compresie} & \textbf{Nr. blocuri} \\
\midrule
30 minute & 3.8x & 1048 \\
1 oră & 4.0x & 524 \\
2 ore (standard) & 4.13x & 262 \\
4 ore & 4.2x & 131 \\
\bottomrule
\end{tabular}
\caption{Impactul dimensiunii blocului}
\label{tab:block-size}
\end{table}

Blocurile mai mari oferă compresie mai bună (contextul XOR se păstrează mai mult), dar:
\begin{itemize}
    \item Cresc latența pentru interogări pe intervale scurte;
    \item Necesită mai multă memorie pentru decodare;
    \item Sunt mai sensibile la corupție (pierderea unui bloc = pierdere mai mare de date).
\end{itemize}

Pragul de 2 ore reprezintă un compromis rezonabil, validat empiric de echipa Facebook.

\section{Validarea corectitudinii}

Toate testele au trecut validarea round-trip:

\begin{lstlisting}[language=Python, caption={Verificare round-trip}]
# Comprima
compressed = series.get_compressed_data()

# Decomprima
decoded = decode_series(compressed, count, var_names)

# Verifica
for original, decoded in zip(original_data, decoded_data):
    assert original.timestamp == decoded.timestamp
    for var in var_names:
        assert original.values[var] == decoded.values[var]

print("[OK] Round-trip validation passed!")
\end{lstlisting}

Toate valorile sunt recuperate exact (bit-perfect), confirmând caracterul \textit{lossless} al compresiei.

\section{Comparație cu articolul original}

\begin{table}[h]
\centering
\begin{tabular}{@{}lcc@{}}
\toprule
\textbf{Metrică} & \textbf{Gorilla (Facebook)} & \textbf{Implementarea noastră} \\
\midrule
Bytes/punct (medie) & 1.37 & 4--7 \\
\% timestamps cu D=0 & 96.39\% & 85\% \\
\% valori XOR=0 & 59.06\% & 45\% \\
Rata compresie & 12x & 2--4x \\
\bottomrule
\end{tabular}
\caption{Comparație cu rezultatele raportate în articolul Gorilla}
\label{tab:comparison}
\end{table}

Diferențele se explică prin:

\begin{enumerate}
    \item \textbf{Caracteristicile datelor}: Datele Facebook provin de la sisteme de monitorizare foarte regulate (interval fix de 60s), în timp ce seturile noastre au mai mult jitter.

    \item \textbf{Tipul valorilor}: Monitorizarea Facebook folosește preponderent contoare (integers convertiți la double), care produc XOR-uri mai compresibile decât măsurătorile analogice de temperatură.

    \item \textbf{Scala}: La 2 miliarde de serii, efectele statistice se netezesc; la câteva sute de puncte, variabilitatea este mai mare.
\end{enumerate}

Cu toate acestea, implementarea demonstrează corectitudinea algoritmului și obține rate de compresie semnificative (2--4x) pe date reale diverse.
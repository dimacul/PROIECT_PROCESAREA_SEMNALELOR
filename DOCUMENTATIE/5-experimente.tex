\chapter{Experimente și Rezultate}

Acest capitol prezintă evaluarea experimentală a implementării algoritmului Gorilla. Sunt descrise seturile de date utilizate, metodologia de testare și rezultatele obținute. Demonstrăm mai întâi eficiența compresiei Gorilla față de stocarea necomprimată, apoi prezentăm îmbunătățirea adusă de varianta optimizată propusă în această lucrare.

\section{Seturi de date}

Pentru validarea implementării au fost utilizate trei seturi de date cu caracteristici diferite: două serii univariate și una multivariată.

\subsection{CPU Load Average (serie univariată)}

Primul set de date conține metrici de încărcare CPU colectate de pe un sistem Linux.

\begin{table}[h]
\centering
\begin{tabular}{@{}ll@{}}
\toprule
\textbf{Caracteristică} & \textbf{Valoare} \\
\midrule
Număr de puncte & 480 \\
Perioadă de colectare & $\sim$16 ore \\
Interval mediu & $\sim$120 secunde \\
Tip date & Univariat (load average) \\
Interval valori & 0.57 -- 2.55 \\
\bottomrule
\end{tabular}
\caption{Caracteristicile setului de date CPU Load}
\label{tab:cpu-dataset}
\end{table}

\textbf{Load average} reprezintă numărul mediu de procese în coadă de execuție pe o fereastră de timp. Valorile variază ușor în timp, cu schimbări graduale caracteristice sarcinilor tipice de server.

\subsection{Twitter Volume UPS (serie univariată)}

Al doilea set de date provine din arhiva Numenta Anomaly Benchmark și conține volumul de tweet-uri care menționează compania UPS.

\begin{table}[h]
\centering
\begin{tabular}{@{}ll@{}}
\toprule
\textbf{Caracteristică} & \textbf{Valoare} \\
\midrule
Număr de puncte & 15,866 \\
Perioadă de colectare & $\sim$55 zile \\
Interval mediu & 5 minute (constant) \\
Tip date & Univariat (volum tweet-uri) \\
Interval valori & 0 -- 2,523 \\
\bottomrule
\end{tabular}
\caption{Caracteristicile setului de date Twitter Volume}
\label{tab:twitter-dataset}
\end{table}

Acest set de date este deosebit de relevant deoarece:
\begin{itemize}
    \item \textbf{Timestamp-uri perfect periodice}: Interval exact de 5 minute, ideal pentru compresia delta-of-delta
    \item \textbf{Valori întregi}: Contoare de tweet-uri, care produc pattern-uri XOR diferite față de valorile continue
    \item \textbf{Dimensiune medie}: Cu 15,866 puncte, permite evaluarea scalabilității
\end{itemize}

\subsection{Room Climate Dataset (serie multivariată)}

Al treilea set de date provine din Room Climate Dataset, o colecție de măsurători de la senzori de mediu interior \cite{roomclimate}.

\begin{table}[h]
\centering
\begin{tabular}{@{}ll@{}}
\toprule
\textbf{Caracteristică} & \textbf{Valoare} \\
\midrule
Număr de puncte & 68,229 \\
Perioadă de colectare & $\sim$16 zile \\
Interval mediu & $\sim$4 secunde \\
Număr variabile & 8 \\
Node ID filtrat & NID = 1 \\
\bottomrule
\end{tabular}
\caption{Caracteristicile setului de date Room Climate}
\label{tab:room-dataset}
\end{table}

Cele 8 variabile măsurate sunt: temperatură, umiditate relativă, lumină 1, lumină 2, ocupare, activitate, stare ușă și stare fereastră.

\section{Metodologia experimentală}

\subsection{Metrici de evaluare}

Au fost calculate următoarele metrici:

\begin{enumerate}
    \item \textbf{Dimensiune necomprimată}: $n \times (8 + 8k)$ bytes, unde $n$ este numărul de puncte și $k$ numărul de variabile. Aceasta corespunde stocării naive: 8 bytes pentru timestamp (int64) + 8 bytes per valoare (float64).

    \item \textbf{Dimensiune comprimată}: numărul de bytes după aplicarea algoritmului Gorilla.

    \item \textbf{Rata de compresie}:
    \begin{equation}
        R = \frac{\text{Dimensiune necomprimată}}{\text{Dimensiune comprimată}}
    \end{equation}

    \item \textbf{Îmbunătățire procentuală} (economie de spațiu):
    \begin{equation}
        E = \left(1 - \frac{\text{Dimensiune comprimată}}{\text{Dimensiune originală}}\right) \times 100\%
    \end{equation}
\end{enumerate}

\subsection{Configurația testelor}

\begin{itemize}
    \item Blocuri de 2 ore (7.200.000 ms), conform configurației standard Gorilla
    \item Validare round-trip pentru fiecare test (compresie $\rightarrow$ decompresie $\rightarrow$ verificare bit-perfect)
\end{itemize}

\section{Rezultate: Tabel comparativ principal}

\subsection{Comparație globală}

\begin{table}[h]
\centering
\small
\begin{tabular}{@{}lccccc@{}}
\toprule
\textbf{Dataset} & \textbf{Necmp.} & \textbf{Gorilla Std} & \textbf{Gorilla Opt} & \textbf{Îmbun. Necmp$\rightarrow$Std} & \textbf{Îmbun. Std$\rightarrow$Opt} \\
\midrule
CPU Load & 7,680 B & 3,485 B & 3,236 B & 54.6\% & 7.14\% \\
Twitter Vol. & 253,856 B & 38,520 B & 38,258 B & 84.8\% & 0.68\% \\
Room Climate & 4,912,488 B & 1,190,271 B & 1,169,018 B & 75.8\% & 1.79\% \\
\midrule
\textbf{TOTAL} & 5,174,024 B & 1,232,276 B & 1,210,512 B & \textbf{76.2\%} & \textbf{1.77\%} \\
\bottomrule
\end{tabular}
\caption{Comparație completă: Necomprimat vs Gorilla Standard vs Gorilla Optimizat}
\label{tab:full-comparison}
\end{table}

\textbf{Interpretare}:
\begin{itemize}
    \item Coloana \textbf{Îmbun. Necmp$\rightarrow$Std}: Cât de mult reduce Gorilla Standard față de stocarea necomprimată
    \item Coloana \textbf{Îmbun. Std$\rightarrow$Opt}: Cât de mult îmbunătățește varianta noastră optimizată față de Gorilla Standard
\end{itemize}

\section{Rezultate detaliate: CPU Load}

\begin{table}[h]
\centering
\begin{tabular}{@{}lcccc@{}}
\toprule
\textbf{Metrică} & \textbf{Necmp.} & \textbf{Gorilla Std} & \textbf{Gorilla Opt} & \textbf{Îmbunătățiri} \\
\midrule
Dimensiune & 7,680 B & 3,485 B & 3,236 B & -- \\
Bytes per punct & 16.00 & 7.26 & 6.74 & -- \\
Biți per punct & 128.00 & 58.10 & 53.95 & -- \\
Rata compresie & 1.00x & 2.20x & 2.37x & -- \\
\midrule
\multicolumn{4}{@{}l}{\textbf{Îmbunătățire Necomprimat $\rightarrow$ Gorilla Standard}} & \textbf{54.6\%} \\
\multicolumn{4}{@{}l}{\textbf{Îmbunătățire Gorilla Standard $\rightarrow$ Gorilla Optimizat}} & \textbf{7.14\%} \\
\bottomrule
\end{tabular}
\caption{CPU Load: comparație detaliată cu îmbunătățiri}
\label{tab:cpu-all}
\end{table}

\textbf{Analiza}: CPU Load beneficiază cel mai mult de optimizare (\textbf{7.14\%}). Valorile load average variază gradual, producând multe situații în care fereastra anterioară era semnificativ mai mare decât necesarul. Economie suplimentară: \textbf{249 bytes}.

\section{Rezultate detaliate: Twitter Volume}

\begin{table}[h]
\centering
\begin{tabular}{@{}lcccc@{}}
\toprule
\textbf{Metrică} & \textbf{Necmp.} & \textbf{Gorilla Std} & \textbf{Gorilla Opt} & \textbf{Îmbunătățiri} \\
\midrule
Dimensiune & 253,856 B & 38,520 B & 38,258 B & -- \\
Bytes per punct & 16.00 & 2.43 & 2.41 & -- \\
Biți per punct & 128.00 & 19.42 & 19.29 & -- \\
Rata compresie & 1.00x & 6.59x & 6.64x & -- \\
\midrule
\multicolumn{4}{@{}l}{\textbf{Îmbunătățire Necomprimat $\rightarrow$ Gorilla Standard}} & \textbf{84.8\%} \\
\multicolumn{4}{@{}l}{\textbf{Îmbunătățire Gorilla Standard $\rightarrow$ Gorilla Optimizat}} & \textbf{0.68\%} \\
\bottomrule
\end{tabular}
\caption{Twitter Volume: comparație detaliată cu îmbunătățiri}
\label{tab:twitter-all}
\end{table}

\textbf{Analiza}: Acest set demonstrează cele mai bune rezultate pentru Gorilla (\textbf{84.8\%} reducere) datorită timestamp-urilor perfect periodice. Îmbunătățirea optimizării este mai modestă (0.68\%) deoarece valorile întregi produc XOR-uri cu pattern-uri mai previzibile. Economie suplimentară: \textbf{262 bytes}.

\section{Rezultate detaliate: Room Climate}

\begin{table}[h]
\centering
\begin{tabular}{@{}lcccc@{}}
\toprule
\textbf{Metrică} & \textbf{Necmp.} & \textbf{Gorilla Std} & \textbf{Gorilla Opt} & \textbf{Îmbunătățiri} \\
\midrule
Dimensiune & 4,912,488 B & 1,190,271 B & 1,169,018 B & -- \\
Bytes per punct & 72.00 & 17.45 & 17.13 & -- \\
Biți per punct & 576.00 & 139.55 & 137.06 & -- \\
Rata compresie & 1.00x & 4.13x & 4.20x & -- \\
\midrule
\multicolumn{4}{@{}l}{\textbf{Îmbunătățire Necomprimat $\rightarrow$ Gorilla Standard}} & \textbf{75.8\%} \\
\multicolumn{4}{@{}l}{\textbf{Îmbunătățire Gorilla Standard $\rightarrow$ Gorilla Optimizat}} & \textbf{1.79\%} \\
\bottomrule
\end{tabular}
\caption{Room Climate: comparație detaliată cu îmbunătățiri}
\label{tab:room-all}
\end{table}

\textbf{Analiza}: Cu 8 variabile și 68,229 puncte, acest set demonstrează scalabilitatea atât a algoritmului Gorilla (economie de \textbf{3.7 MB}), cât și a optimizării noastre (economie suplimentară de \textbf{21,253 bytes} $\approx$ 20.7 KB).

\section{Sumar rate de compresie}

\begin{table}[h]
\centering
\begin{tabular}{@{}lcccc@{}}
\toprule
\textbf{Dataset} & \textbf{Rata Std} & \textbf{Rata Opt} & \textbf{Îmbun. Necmp$\rightarrow$Std} & \textbf{Îmbun. Std$\rightarrow$Opt} \\
\midrule
CPU Load & 2.20x & 2.37x & 54.6\% & 7.14\% \\
Twitter Volume & 6.59x & 6.64x & 84.8\% & 0.68\% \\
Room Climate & 4.13x & 4.20x & 75.8\% & 1.79\% \\
\midrule
\textbf{Media} & 4.31x & 4.40x & \textbf{71.7\%} & \textbf{3.20\%} \\
\bottomrule
\end{tabular}
\caption{Sumar rate de compresie și îmbunătățiri}
\label{tab:compression-rates}
\end{table}

\section{Varianta optimizată: Recapitulare}

\subsection{Condiția standard vs optimizată}

Varianta standard Gorilla refolosește fereastra XOR anterioară dacă:
\begin{equation}
    L_{cur} \geq L_{ant} \quad \land \quad T_{cur} \geq T_{ant}
\end{equation}

Varianta noastră adaugă o condiție suplimentară:
\begin{equation}
    L_{cur} \geq L_{ant} \quad \land \quad T_{cur} \geq T_{ant} \quad \land \quad (M_{ant} - M_{cur} \leq 11)
\end{equation}

unde $M = 64 - L - T$ este numărul de biți semnificativi. Această condiție evită refolosirea ineficientă a unei ferestre prea mari.

\subsection{Exemplu numeric}

Fie un scenariu concret:
\begin{itemize}
    \item Valoarea anterioară: fereastră de 45 biți semnificativi ($L = 10, T = 9$)
    \item Valoarea curentă: necesită doar 8 biți semnificativi ($L = 28, T = 28$)
\end{itemize}

\begin{table}[h]
\centering
\begin{tabular}{@{}lccc@{}}
\toprule
\textbf{Variantă} & \textbf{Decizie} & \textbf{Calcul} & \textbf{Cost} \\
\midrule
Standard & Refolosire & $2 + 45$ & 47 biți \\
Optimizat & Fereastră nouă & $2 + 5 + 6 + 8$ & 21 biți \\
\midrule
\multicolumn{3}{@{}l}{\textbf{Economie per punct}} & \textbf{26 biți} \\
\bottomrule
\end{tabular}
\caption{Comparație cost pentru un singur punct}
\label{tab:cost-comparison}
\end{table}

\section{Comparație cu articolul original Gorilla}

\begin{table}[h]
\centering
\begin{tabular}{@{}lccc@{}}
\toprule
\textbf{Metrică} & \textbf{Facebook} & \textbf{Impl. Std} & \textbf{Impl. Opt} \\
\midrule
Bytes/punct & 1.37 & 2.4--7.3 & 2.4--6.7 \\
\% timestamps D=0 & 96.4\% & 85\%--99\% & 85\%--99\% \\
\% valori XOR=0 & 59.1\% & $\sim$45\% & $\sim$45\% \\
Rata compresie & 12x & 2.2--6.6x & 2.4--6.6x \\
\bottomrule
\end{tabular}
\caption{Comparație cu rezultatele raportate în articolul Gorilla}
\label{tab:comparison-original}
\end{table}

Diferențele se explică prin caracteristicile datelor: Facebook folosește metrici de monitorizare foarte regulate, în timp ce seturile noastre au mai multă variabilitate.

\section{Validarea corectitudinii}

Toate testele au trecut validarea round-trip, confirmând caracterul \textit{lossless}:

\begin{lstlisting}[language=Python, caption={Verificare round-trip}]
# Comprima
for ts, vals in original_data:
    series.insert(ts, vals)
series.flush()

# Decomprima
decoder = MultiVariateDecoder(compressed_data, var_names)
decoded = [decoder.read_point() for _ in range(count)]

# Verifica (bit-perfect)
for (ts_orig, vals_orig), (ts_dec, vals_dec) in zip(original, decoded):
    assert ts_orig == ts_dec
    for var in var_names:
        assert vals_orig[var] == vals_dec[var]
\end{lstlisting}

\section{Concluzii}

\begin{table}[h]
\centering
\begin{tabular}{@{}lcc@{}}
\toprule
\textbf{Concluzie} & \textbf{Îmbun. Necmp$\rightarrow$Std} & \textbf{Îmbun. Std$\rightarrow$Opt} \\
\midrule
CPU Load & 54.6\% & 7.14\% \\
Twitter Volume & 84.8\% & 0.68\% \\
Room Climate & 75.8\% & 1.79\% \\
\midrule
\textbf{Total/Medie} & \textbf{76.2\%} & \textbf{1.77\%} \\
\bottomrule
\end{tabular}
\caption{Sumar final al îmbunătățirilor}
\label{tab:final-summary}
\end{table}

Experimentele demonstrează:

\begin{enumerate}
    \item \textbf{Eficiența Gorilla}: Algoritmul reduce dimensiunea datelor cu \textbf{55--85\%} față de stocarea necomprimată, confirmând valoarea sa pentru serii temporale.

    \item \textbf{Optimizarea funcționează}: Pe toate cele 3 seturi de date, varianta optimizată produce fișiere mai mici decât cea standard, cu îmbunătățiri între \textbf{0.68\% și 7.14\%}.

    \item \textbf{Fără compromisuri}: Nu există niciun caz în care varianta standard ar fi mai bună --- optimizarea este \textbf{universal benefică}.

    \item \textbf{Scalabilitate}: Cu cât mai multe date, cu atât economia absolută este mai mare (21+ KB pentru Room Climate).
\end{enumerate}

\textbf{Recomandare}: Varianta optimizată ar trebui folosită întotdeauna în locul celei standard, fiind superioară din punct de vedere matematic și confirmat experimental.

\section{Anexă: Vizualizări grafice}

Această secțiune prezintă vizualizări grafice ale seturilor de date și rezultatelor experimentale.

\subsection{Seria CPU Load}

\begin{figure}[H]
\centering
\includegraphics[width=0.95\textwidth]{../grafice_output/grafic_cpu_load.pdf}
\caption{Vizualizarea seriei temporale CPU Load Average. Se observă variația graduală a valorilor load average pe parcursul perioadei de colectare.}
\label{fig:cpu-load-graph}
\end{figure}

\subsection{Seria Room Climate}

\begin{figure}[H]
\centering
\includegraphics[width=0.95\textwidth]{../grafice_output/grafic_room_climate.pdf}
\caption{Vizualizarea seriei multivariate Room Climate. Graficul prezintă evoluția celor 8 variabile măsurate de senzorii de mediu interior.}
\label{fig:room-climate-graph}
\end{figure}

\subsection{Seria Twitter Volume --- Compresia Standard}

\begin{figure}[H]
\centering
\includegraphics[width=0.95\textwidth]{../grafice_output/grafic_twitter_standard.pdf}
\caption{Vizualizarea seriei Twitter Volume procesată cu algoritmul Gorilla Standard. Graficul arată volumul de tweet-uri pe parcursul perioadei de colectare.}
\label{fig:twitter-standard-graph}
\end{figure}

\subsection{Seria Twitter Volume --- Compresia Optimizată}

\begin{figure}[H]
\centering
\includegraphics[width=0.95\textwidth]{../grafice_output/grafic_twitter_verificare.pdf}
\caption{Vizualizarea seriei Twitter Volume procesată cu varianta Gorilla Optimizată. Datele decomprimate sunt identice cu cele din varianta standard, confirmând corectitudinea implementării.}
\label{fig:twitter-verificare-graph}
\end{figure}

\subsection{Validarea integrității datelor --- Twitter}

\begin{figure}[H]
\centering
\includegraphics[width=0.95\textwidth]{../grafice_output/comparatie_twitter_integritate.pdf}
\caption{Comparația integrității datelor pentru seria Twitter Volume. Graficul suprapune datele originale cu cele decomprimate din ambele variante (Standard și Optimizată), demonstrând că compresia este \textit{lossless} --- toate valorile sunt recuperate exact.}
\label{fig:twitter-integrity-graph}
\end{figure}
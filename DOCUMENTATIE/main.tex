% Documentație pentru proiectul de compresie Gorilla
% Curs: Procesarea Semnalelor
%
% Bazat pe șablonul FMI UniBuc (Gabriel Majeri)
% Licențiat sub Creative Commons Attribution 4.0 International License.

\documentclass[12pt, a4paper]{report}

% Suport pentru diacritice și alte simboluri
\usepackage{fontspec}

% Font Times New Roman (pentru XeLaTeX)
\setmainfont{Times New Roman}

% Suport pentru mai multe limbi
\usepackage{polyglossia}

% Setează limba textului la română
\setdefaultlanguage{romanian}
% Am nevoie de engleză pentru rezumat
\setotherlanguages{english}

% Indentează și primul paragraf al fiecărei noi secțiuni
\SetLanguageKeys{romanian}{indentfirst=true}

% Suport pentru diferite stiluri de ghilimele
\usepackage{csquotes}

\DeclareQuoteStyle{romanian}
  {\quotedblbase}
  {\textquotedblright}
  {\guillemotleft}
  {\guillemotright}

% Utilizează biblatex pentru referințe bibliografice
\usepackage[
    maxbibnames=50,
    sorting=nty,
    backend=bibtex
]{biblatex}

\addbibresource{bibliography.bib}

% Setează spațiere inter-linie
\usepackage{setspace}
\onehalfspacing

% Modificarea geometriei paginii (cu margini de 2,5 cm)
\usepackage[margin=2.5cm]{geometry}

% Include funcțiile de grafică
\usepackage{graphicx}
% Încarcă imaginile din directorul `images`
\graphicspath{{./images/}}

% Listări de cod
\usepackage{listings}
\usepackage{xcolor}

% Configurare listings pentru Python
\lstdefinestyle{python}{
    language=Python,
    basicstyle=\ttfamily\small,
    keywordstyle=\color{blue}\bfseries,
    stringstyle=\color{red},
    commentstyle=\color{gray}\itshape,
    numbers=left,
    numberstyle=\tiny\color{gray},
    stepnumber=1,
    numbersep=8pt,
    backgroundcolor=\color{gray!5},
    frame=single,
    framerule=0.5pt,
    rulecolor=\color{gray!50},
    breaklines=true,
    breakatwhitespace=true,
    showstringspaces=false,
    tabsize=4,
    captionpos=b,
    xleftmargin=15pt,
    framexleftmargin=15pt,
}

\lstset{style=python}

% Linkuri interactive în PDF
\usepackage[
    colorlinks,
    linkcolor={blue!70!black},
    menucolor={black},
    citecolor={blue!70!black},
    urlcolor={blue}
]{hyperref}

% Comenzi matematice
\usepackage{amsmath}
\usepackage{amssymb}
\usepackage{mathtools}

% Formule matematice
\newcommand{\bigO}[1]{\mathcal{O}\left(#1\right)}
\DeclarePairedDelimiter\abs{\lvert}{\rvert}

% Tabele mai frumoase
\usepackage{booktabs}
\usepackage{array}
\usepackage{multirow}

% Algoritmi
\usepackage{algorithm}
\usepackage{algpseudocode}

% Pentru grafice și diagrame
\usepackage{tikz}
\usetikzlibrary{shapes,arrows,positioning,calc,fit,backgrounds}

% Suport pentru subcaptions
\usepackage{subcaption}

% Liste compacte
\usepackage{enumitem}
\setlist{nosep, topsep=0pt}

% Suport pentru rezumat în două limbi
\newenvironment{abstractpage}
  {\cleardoublepage\vspace*{\fill}\thispagestyle{empty}}
  {\vfill\cleardoublepage}
\renewenvironment{abstract}[1]
  {\bigskip\selectlanguage{#1}%
   \begin{center}\bfseries\abstractname\end{center}}
  {\par\bigskip}

% Suport pentru anexe
\usepackage{appendix}

% Definire mediu example
\usepackage{amsthm}
\theoremstyle{definition}
\newtheorem{example}{Exemplu}[chapter]

% Stiluri diferite de headere și footere
\usepackage{fancyhdr}

% Metadate
\title{Compresia Seriilor De Timp folosind Algoritmul Gorilla}
\author{Student}

% Generează variabilele cu @
\makeatletter

\begin{document}

% Front matter
\cleardoublepage
\let\ps@plain

% Pagina de titlu
\begin{titlepage}

% Redu marginile
\newgeometry{left=2cm,right=2cm,bottom=1cm}

\begin{figure}[!htb]
    \centering
    \begin{minipage}{0.2\textwidth}
        \includegraphics[width=\linewidth]{logo-ub.png}
    \end{minipage}
    \begin{minipage}{0.5\textwidth}
        \large
        \vspace{0.2cm}
        \begin{center}
            \textbf{UNIVERSITATEA DIN BUCUREȘTI}
        \end{center}
        \vspace{0.3cm}
        \begin{center}
            \textbf{
                FACULTATEA DE \\
                MATEMATICĂ ȘI INFORMATICĂ
            }
        \end{center}
    \end{minipage}
    \begin{minipage}{0.2\textwidth}
        \includegraphics[width=\linewidth]{logo-fmi.png}
    \end{minipage}
\end{figure}

\begin{center}
\textbf{SPECIALIZAREA INFORMATICĂ}
\end{center}

\vspace{0.5cm}

\begin{center}
\Large \textbf{Proiect --- Procesarea Semnalelor}
\end{center}

\vspace{0.5cm}

\begin{center}
\huge \textbf{\MakeUppercase{Compresia Seriilor de Timp folosind Algoritmul Gorilla}}
\end{center}

\vspace{2cm}

\begin{center}
\large \textbf{Dima Cristian-Mihai}
\end{center}

\begin{center}
\large \textbf{Muscalu David-Cristian}
\end{center}

\vspace{0.25cm}

\begin{center}
\large \textbf{Coordonator științific \\ Prof. Dr. Cristian Rusu}
\end{center}

\vspace{2cm}

\begin{center}
\Large \textbf{București, 2026}
\end{center}
\end{titlepage}
\restoregeometry
\newgeometry{
    margin=2.5cm
}

\fancypagestyle{main}{
  \fancyhf{}
  \renewcommand\headrulewidth{0pt}
  \fancyhead[C]{}
  \fancyfoot[C]{\thepage}
}

\addtocounter{page}{1}

% Rezumatul
\begin{abstractpage}

\begin{abstract}{romanian}
Sistemele moderne de monitorizare generează volume imense de date sub formă de serii de timp --- secvențe ordonate 
de perechi \textit{(timestamp, value)} care înregistrează evoluția în timp a diferitelor metrici. Stocarea eficientă 
a acestor date reprezintă o provocare semnificativă, deoarece abordarea naivă necesită 16 bytes pentru fiecare datapoint 
(8 bytes pentru timestamp și 8 pentru valoarea în format IEEE 754 double).

În acest proiect analizăm și implementăm algoritmul de compresie \textbf{Gorilla}, dezvoltat de Facebook pentru baza lor de date de serii de timp in-memory. 
Soluția propusă combină două tehnici complementare: \textit{delta-of-delta encoding} pentru comprimarea timestamp-urilor 
și \textit{XOR encoding} pentru comprimarea valorilor în virgulă mobilă. 
Ambele tehnici exploatează proprietățile specifice ale seriilor temporale --- periodicitatea timestamp-urilor și 
corelația temporală a valorilor --- pentru a obține rate de compresie semnificative.

Implementarea realizată de noi în Python demonstrează eficiența algoritmului pe două seturi de date: metrici de utilizare CPU (serie univariată) și 
date climatice dintr-un mediu interior (serie multivariată cu 8 variabile). 
Rezultatele experimentale arată rate de compresie de aproximativ 2--4x, reducând consumul de memorie de la 16 bytes la aproximativ 4--8 bytes per datapoint, 
în funcție de caracteristicile datelor.
\end{abstract}



\end{abstractpage}

\tableofcontents

% Main matter
\cleardoublepage
\pagestyle{main}
\let\ps@plain\ps@main

\chapter{Introducere}

\section{Contextul problemei}

O \textbf{serie de timp} este o secvență ordonată de observații înregistrate la momente succesive de timp, de obicei la intervale regulate. Formal, o serie temporală poate fi definită ca o funcție:
\begin{equation}
    X: T \rightarrow \mathbb{R}, \quad X = \{(t_1, x_1), (t_2, x_2), \ldots, (t_n, x_n)\}
\end{equation}
unde $T \subseteq \mathbb{Z}$ reprezintă mulțimea timestamp-urilor (de regulă, în milisecunde sau secunde de la epoca Unix), iar $x_i \in \mathbb{R}$ reprezintă valoarea observată la momentul $t_i$.

Aplicațiile care generează serii de timp sunt omniprezente în infrastructura digitală contemporană, iar
volumul de date generate de aceste sisteme este enorm. Pentru a ilustra scala problemei, 
să considerăm sistemul de monitorizare al companiei Facebook (Meta), descris în articolul original Gorilla \cite{gorilla2015}:
\begin{itemize}
    \item Peste \textbf{2 miliarde} de serii temporale unice;
    \item Aproximativ \textbf{12 milioane} de datapoints adăugate \textit{în fiecare secundă};
    \item Aceasta echivalează cu peste \textbf{1 trilion} de puncte pe zi;
    \item La 16 bytes per datapoint (8 pentru timestamp + 8 pentru valoare), rezultă un necesar de aproximativ \textbf{16 TB de RAM} zilnic.
\end{itemize}

Această rată masivă de ingestie a datelor face imposibilă stocarea necomprimată și impune dezvoltarea unor tehnici eficiente de compresie.

\section{Motivația proiectului}

Stocarea seriilor temporale prezintă provocări specifice care diferă de cele întâlnite în bazele de date tradiționale:

\subsection{Cerințe de performanță}

Sistemele de monitorizare necesită:
\begin{enumerate}
    \item \textbf{Disponibilitate ridicată pentru scrieri}: Datele trebuie să poată fi scrise în orice moment, chiar și în prezența unor defecțiuni parțiale ale sistemului. O rată ridicată de succes pentru operațiile de scriere este esențială, deoarece pierderea datelor de monitorizare poate masca probleme critice.

    \item \textbf{Latență redusă pentru citiri}: Interogările trebuie să returneze rezultate în milisecunde, nu secunde. Aceasta permite inginerilor să reacționeze rapid la anomalii și să diagnosticheze probleme în timp real.

    \item \textbf{Prioritizarea datelor recente}: În contextul monitorizării, datele recente sunt mult mai valoroase decât cele vechi. Faptul că un serviciu a fost indisponibil acum 5 minute este mai relevant decât același eveniment petrecut acum o săptămână.
\end{enumerate}

\subsection{Particularități ale datelor}

Seriile temporale de monitorizare au proprietăți distincte care pot fi exploatate pentru compresie:

\begin{enumerate}
    \item \textbf{Timestamp-uri cvasi-periodice}: Majoritatea sistemelor de monitorizare colectează date la intervale regulate (de exemplu, la fiecare 60 de secunde). Deși pot exista mici variații (jitter) datorită latențelor de rețea sau încărcării sistemului, intervalul dintre timestamp-uri consecutive rămâne relativ constant.

    \item \textbf{Corelație temporală a valorilor}: Valorile consecutive dintr-o serie temporală tind să fie similare. Temperatura într-o cameră nu sare brusc de la 22°C la 50°C; utilizarea CPU a unui server nu variază dramatic de la o secundă la alta în condiții normale.

    \item \textbf{Toleranță la pierderi minore}: Spre deosebire de sistemele financiare sau medicale, sistemele de monitorizare pot tolera pierderea ocazională a câtorva puncte de date, atâta timp cât tendințele generale rămân vizibile.
\end{enumerate}

\section{Obiectivele proiectului}

Prezentul proiect își propune:

\begin{enumerate}
    \item \textbf{Înțelegerea și implementarea algoritmului Gorilla}: Studiul detaliat al tehnicilor de compresie delta-of-delta și XOR, precum și implementarea lor completă în Python.

    \item \textbf{Suport pentru serii multivariate}: Extinderea algoritmului pentru a gestiona eficient serii cu mai multe variabile per timestamp.

    \item \textbf{Validarea experimentală}: Testarea implementării pe seturi de date reale și măsurarea ratelor de compresie obținute, comparativ cu stocarea naivă.

    \item \textbf{Documentarea completă}: Prezentarea detaliată a fundamentelor teoretice, a algoritmilor și a deciziilor de implementare.
\end{enumerate}
\chapter{State-of-the-Art}

Înainte de a prezenta algoritmul Gorilla, vom trece în revistă provocările cărora bazele de date cu serii de timp încearcă să le facă față și compromisurile pe care le fac.

\section{Baze de date pentru serii de timp (TSDB)}

O \textbf{bază de date pentru serii temporale} (Time Series Database --- TSDB) este un sistem software optimizat pentru stocarea și interogarea datelor indexate după timp. Spre deosebire de bazele de date relaționale tradiționale, TSDB-urile sunt proiectate pentru:

\begin{itemize}
    \item Rate foarte mari de scriere (milioane de puncte pe secundă);
    \item Interogări pe intervale de timp (,,dă-mi toate valorile între ora 10:00 și 11:00'');
    \item Agregări temporale (medii, sume, percentile pe ferestre de timp);
    \item Retenție automată a datelor (ștergerea datelor mai vechi de X zile).
\end{itemize}

\section{OpenTSDB}

\textbf{OpenTSDB} (Open Time Series Database) este o bază de date distribuită pentru serii temporale, construită peste Apache HBase \cite{opentsdb}. HBase, la rândul său, este un sistem de stocare distribuit bazat pe modelul BigTable al Google.

\subsection{Arhitectură}

OpenTSDB utilizează un model de date în care fiecare serie temporală este identificată printr-un set de perechi cheie-valoare numite \textit{tag-uri}:

\begin{verbatim}
    metric: sys.cpu.user
    tags: {host=webserver01, cpu=0}
    timestamp: 1458034800
    value: 42.5
\end{verbatim}

Datele sunt stocate în HBase într-o structură de tabel optimizată pentru scanări secvențiale pe intervale de timp.

\subsection{Avantaje și limitări}

\textbf{Avantaje:}
\begin{itemize}
    \item Scalabilitate orizontală prin adăugarea de noduri HBase;
    \item Persistență durabilă pe disc cu replicare;
    \item Model de date flexibil cu tag-uri arbitrare.
\end{itemize}

\textbf{Limitări:}
\begin{itemize}
    \item \textbf{Latență ridicată}: Fiind bazat pe disc, interogările pot dura secunde sau chiar minute pentru volume mari de date;
    \item \textbf{Complexitate operațională}: Necesită administrarea unui cluster HBase și Hadoop;
    \item \textbf{Nu este optimizat pentru date recente}: Toate datele sunt tratate egal, fără prioritizare a celor recente.
\end{itemize}

În contextul Facebook, OpenTSDB (prin HBase) nu putea scala pentru cerințele de latență necesare. Percentila 90 a timpului de interogare creștea de la câteva secunde la minute pe măsură ce volumul de date creștea.

\section{Graphite și Whisper}

\textbf{Graphite} este o suită de monitorizare compusă din trei componente: Carbon (serviciul de ingestie), Whisper (formatul de stocare) și Graphite Web (interfața de interogare) \cite{graphite}.

\subsection{Formatul Whisper}

Whisper utilizează un format de stocare \textit{Round-Robin Database} (RRD). Caracteristicile principale:

\begin{itemize}
    \item \textbf{Dimensiune fixă}: Fiecare serie temporală ocupă un spațiu fix pe disc, determinat la creare;
    \item \textbf{Intervale fixe}: Timestamp-urile sunt presupuse să vină la intervale exacte;
    \item \textbf{Suprascrierea datelor vechi}: Când spațiul se umple, datele noi le suprascriu pe cele mai vechi.
\end{itemize}

\subsection{Limitări}

\begin{itemize}
    \item \textbf{Nu suportă jitter}: Dacă un punct de date vine la secunda 61 în loc de 60, trebuie fie ignorat, fie interpolat;
    \item \textbf{Stocare pe disc}: Similar cu OpenTSDB, latența este limitată de I/O-ul discului;
    \item \textbf{Fiecare serie într-un fișier separat}: La miliarde de serii, managementul fișierelor devine problematic.
\end{itemize}

\section{Tehnici de compresie pentru serii de timp}

Există multe abordări pentru compresia seriilor de timp, fiecare cu propriile dezavantaje:

\subsection{Compresie cu pierderi (Lossy)}

Tehnicile cu pierderi reduc volumul de date prin aproximare:

\begin{itemize}
    \item \textbf{Downsampling}: Păstrarea doar a unui punct din N (de exemplu, un punct pe minut în loc de unul pe secundă);
    \item \textbf{Agregare}: Înlocuirea mai multor puncte cu media, suma sau alte statistici;
    \item \textbf{Piecewise Linear Approximation}: Aproximarea seriei cu segmente de dreaptă.
\end{itemize}

Aceste tehnici sunt utile pentru date istorice, dar inacceptabile pentru monitorizare în timp real, unde fiecare punct poate indica o problemă critică.

\subsection{Compresie fără pierderi (Lossless)}

Tehnicile fără pierderi păstrează exact datele originale:

\begin{itemize}
    \item \textbf{Delta encoding}: Stocarea diferențelor între valori consecutive în loc de valorile absolute;
    \item \textbf{Run-length encoding (RLE)}: Compresia secvențelor de valori identice;
\end{itemize}

\subsection{Compresie pentru numere în virgulă mobilă}

Compresia valorilor \textit{floating-point} prezintă următoarele provocări:

\begin{itemize}
    \item Reprezentarea IEEE 754 distribuie informația pe 64 de biți (semn, exponent, mantisă);
    \item Diferențele mici între valori pot produce reprezentări binare foarte diferite;
\end{itemize}

Lindstrom și Isenburg \cite{lindstrom2006} au propus o tehnică bazată pe predicție și codare XOR pentru compresia datelor științifice în virgulă mobilă. Gorilla adaptează această tehnică pentru streaming în timp real.

\section{Poziționarea algoritmului Gorilla}

Gorilla ocupă o nișă distinctă în peisajul TSDB-urilor:

\begin{table}[h]
\centering
\begin{tabular}{@{}lccc@{}}
\toprule
\textbf{Caracteristică} & \textbf{OpenTSDB} & \textbf{Graphite} & \textbf{Gorilla} \\
\midrule
Stocare & Disc (HBase) & Disc (Whisper) & Memorie \\
Latență citire & Secunde & Secunde & Milisecunde \\
Compresie & Minimă & Fără & Delta-of-delta + XOR \\
Jitter timestamp & Da & Nu & Da \\
Prioritizare date recente & Nu & Nu & Da \\
Persistență & Da & Da & Opțional (cache) \\
\bottomrule
\end{tabular}
\caption{Comparație între sisteme TSDB}
\label{tab:tsdb-comparison}
\end{table}

Gorilla este gândit ca un \textbf{cache write-through} în fața unui TSDB persistent precum HBase. 

Un write-through cache este un mecanism unde datele sunt scrise simultan în cache și în mediul de stocare permanent (back-end storage).
Când o aplicație trimite un punct de date (un datapoint), acesta este stocat imediat în memoria RAM a lui Gorilla pentru interogări ultra-rapide și, în paralel, este trimis către un sistem persistent (precum HBase) pentru arhivare pe termen lung.
Cache-ul este mereu „la zi” cu baza de date. Dacă interogarea caută date recente, Gorilla le oferă instantaneu fără a mai accesa discul.

Aceste caracteristici au făcut din Gorilla sistemul preferat pentru monitorizare în timp real la Facebook, servind peste 85\% din interogările către datele din ultimele 26 de ore.
\chapter{Algoritmul Gorilla}
\label{chap:gorilla}

Acest capitol prezintă în detaliu algoritmul de compresie Gorilla, fundamentele sale matematice și schemele de codare utilizate. Algoritmul este compus din două tehnici complementare: \textbf{delta-of-delta encoding} pentru timestamp-uri și \textbf{XOR encoding} pentru valori în virgulă mobilă.

\section{Arhitectura generală}

\subsection{Structura datelor}

În Gorilla, datele sunt organizate în \textbf{blocuri} de durată fixă (implicit 2 ore). Fiecare bloc conține:

\begin{enumerate}
    \item Un \textbf{header} cu timestamp-ul de start al blocului (aliniat la granița de 2 ore);
    \item O secvență de perechi \textit{(timestamp, valoare)} comprimate;
    \item Metadate despre numărul de puncte din bloc.
\end{enumerate}

Alegerea blocurilor de 2 ore reprezintă un compromis între rata de compresie (blocuri mai mari = compresie mai bună) și granularitatea accesului (blocuri mai mici = citire mai rapidă pentru intervale scurte).

\subsection{Fluxul de compresie}

Schema generală de compresie este ilustrată în Figura~\ref{fig:compression-overview}.

\begin{figure}[h]
\centering
\begin{tikzpicture}[
    box/.style={draw, rectangle, minimum width=2.5cm, minimum height=0.8cm, align=center},
    arrow/.style={->, thick}
]
    % Input stream
    \node[box] (input) at (0,0) {Date brute\\$(t_i, v_i)$};

    % Timestamp compression
    \node[box] (ts) at (4,1) {Compresie\\Timestamps};

    % Value compression
    \node[box] (val) at (4,-1) {Compresie\\Valori};

    % Interleaved output
    \node[box] (output) at (8,0) {Flux de biți\\comprimat};

    % Arrows
    \draw[arrow] (input) -- (ts);
    \draw[arrow] (input) -- (val);
    \draw[arrow] (ts) -- (output);
    \draw[arrow] (val) -- (output);

\end{tikzpicture}
\caption{Fluxul general de compresie în Gorilla}
\label{fig:compression-overview}
\end{figure}

\section{Compresia Timestamp-urilor: Delta-of-Delta}
\label{sec:dod}

\subsection{Observația cheie}

Timestamp-urile în sistemele de monitorizare sunt de obicei \textbf{cvasi-periodice}. Dacă un sistem colectează date la fiecare 60 de secunde, timestamp-urile consecutive vor fi:
\begin{equation}
    t_0, t_0 + 60, t_0 + 120, t_0 + 180, \ldots
\end{equation}

Chiar dacă există mici variații (de exemplu, $t_0 + 61$ în loc de $t_0 + 60$), diferența $\delta_i = t_i - t_{i-1}$ rămâne aproape constantă.

\subsection{Delta encoding simplu}

Prima idee ar fi să stocăm diferențele (delta-urile) dintre timestamp-ul curent și cel anterior în loc de valoarea timstamp-ului curent:
\begin{equation}
    \delta_i = t_i - t_{i-1}
\end{equation}

Pentru timestamp-urile de mai sus: $\delta_1 = 60, \delta_2 = 60, \delta_3 = 60, \ldots$

Astfel reducem valorile de stocat de la 64 de biți (timestamp absolut) la câțiva biți (delta mic). Dar putem obține o stocare a datelor și mai eficientă!

\subsection{Delta-of-delta encoding}

Gorilla utilizează \textbf{delta-of-delta} (diferența diferențelor):
\begin{equation}
    D_i = \delta_i - \delta_{i-1} = (t_i - t_{i-1}) - (t_{i-1} - t_{i-2})
\end{equation}

Pentru timestamp-uri perfect periodice, $D_i = 0$ pentru toate $i > 1$. Aceasta este observația crucială: dacă timestamp-urile sunt regulate, \textbf{delta-of-delta este zero}.

\begin{example}
Să considerăm timestamp-urile: $1000, 1060, 1120, 1185, 1245$.

\begin{center}
\begin{tabular}{ccccc}
\toprule
$i$ & $t_i$ & $\delta_i = t_i - t_{i-1}$ & $D_i = \delta_i - \delta_{i-1}$ \\
\midrule
0 & 1000 & --- & --- \\
1 & 1060 & 60 & --- \\
2 & 1120 & 60 & 0 \\
3 & 1185 & 65 & +5 \\
4 & 1245 & 60 & -5 \\
\bottomrule
\end{tabular}
\end{center}

Observăm că $D_2 = 0$ (periodicitate perfectă), iar $D_3$ și $D_4$ sunt valori mici ($\pm 5$) care pot fi stocate eficient.
\end{example}

\subsection{Schema de codare variabilă}

Gorilla utilizează o schemă de codare cu lungime variabilă pentru valorile delta-of-delta, optimizată pentru distribuția observată în date reale:

\begin{table}[h]
\centering
\begin{tabular}{@{}lllc@{}}
\toprule
\textbf{Condiție} & \textbf{Prefix} & \textbf{Valoare} & \textbf{Total biți} \\
\midrule
$D = 0$ & \texttt{0} & --- & 1 \\
$D \in [-64, 63]$ & \texttt{10} & 7 biți signed & 9 \\
$D \in [-256, 255]$ & \texttt{110} & 9 biți signed & 12 \\
$D \in [-2048, 2047]$ & \texttt{1110} & 12 biți signed & 16 \\
Altfel & \texttt{1111} & 32 biți signed & 36 \\
\bottomrule
\end{tabular}
\caption{Schema de codare pentru delta-of-delta}
\label{tab:dod-encoding}
\end{table}

\subsection{Reprezentarea în complement față de 2}

Valorile delta-of-delta pot fi negative, deci trebuie reprezentate ca numere cu semn. Gorilla utilizează reprezentarea în \textbf{two's complement}.

Pentru un număr $x$ reprezentat pe $n$ biți:
\begin{itemize}
    \item Dacă $x \geq 0$: reprezentarea este $x$ în binar;
    \item Dacă $x < 0$: reprezentarea este $2^n + x$.
\end{itemize}

Intervalul reprezentabil pe $n$ biți este $[-2^{n-1}, 2^{n-1} - 1]$.

\begin{example}
Pentru $n = 7$ biți (intervalul $[-64, 63]$):
\begin{itemize}
    \item $x = 5$: reprezentare = $0000101_2$
    \item $x = -5$: reprezentare = $2^7 + (-5) = 128 - 5 = 123 = 1111011_2$
\end{itemize}
\end{example}

\subsection{Algoritmul de codare}

Este prezentat în cadrul \textbf{Algorithm 1}.

\begin{algorithm}[h]
\caption{Codare timestamp cu delta-of-delta}
\label{alg:ts-encode}
\begin{algorithmic}[1]
\Require Timestamp $t_n$, timestamp anterior $t_{n-1}$, delta anterior $\delta_{n-1}$
\Ensure Biți scriși în fluxul de ieșire

\State $\delta_n \gets t_n - t_{n-1}$
\State $D \gets \delta_n - \delta_{n-1}$

\If{$D = 0$}
    \State \textbf{write} \texttt{0} \Comment{1 bit}
\ElsIf{$-64 \leq D \leq 63$}
    \State \textbf{write} \texttt{10} \Comment{2 biți prefix}
    \State \textbf{write} $D$ pe 7 biți în complement față de 2
\ElsIf{$-256 \leq D \leq 255$}
    \State \textbf{write} \texttt{110} \Comment{3 biți prefix}
    \State \textbf{write} $D$ pe 9 biți în complement față de 2
\ElsIf{$-2048 \leq D \leq 2047$}
    \State \textbf{write} \texttt{1110} \Comment{4 biți prefix}
    \State \textbf{write} $D$ pe 12 biți în complement față de 2
\Else
    \State \textbf{write} \texttt{1111} \Comment{4 biți prefix}
    \State \textbf{write} $D$ pe 32 biți în complement față de 2
\EndIf

\State \Return $\delta_n$ \Comment{pentru următoarea iterație}
\end{algorithmic}
\end{algorithm}

\subsection{Algoritmul de decodare}

Decodarea citește biții de control pentru a determina lungimea valorii. Vezi \textbf{Algorithm 2}.

\begin{algorithm}[h]
\caption{Decodare timestamp}
\label{alg:ts-decode}
\begin{algorithmic}[1]
\Require Flux de biți, timestamp anterior $t_{n-1}$, delta anterior $\delta_{n-1}$
\Ensure Timestamp decodat $t_n$

\State $b_1 \gets$ \textbf{read} 1 bit
\If{$b_1 = 0$}
    \State $D \gets 0$
\Else
    \State $b_2 \gets$ \textbf{read} 1 bit
    \If{$b_2 = 0$}
        \State $D \gets$ \textbf{read} 7 biți signed
    \Else
        \State $b_3 \gets$ \textbf{read} 1 bit
        \If{$b_3 = 0$}
            \State $D \gets$ \textbf{read} 9 biți signed
        \Else
            \State $b_4 \gets$ \textbf{read} 1 bit
            \If{$b_4 = 0$}
                \State $D \gets$ \textbf{read} 12 biți signed
            \Else
                \State $D \gets$ \textbf{read} 32 biți signed
            \EndIf
        \EndIf
    \EndIf
\EndIf

\State $\delta_n \gets \delta_{n-1} + D$
\State $t_n \gets t_{n-1} + \delta_n$
\State \Return $t_n$
\end{algorithmic}
\end{algorithm}

\subsection{Eficiența compresiei timestamp-urilor}

În datele reale de la Facebook, distribuția valorilor delta-of-delta este puternic concentrată în jurul lui zero:

\begin{itemize}
    \item \textbf{96.39\%} dintre timestamp-uri au $D = 0$ (comprimate la 1 bit);
    \item \textbf{3.35\%} au $D \in [-64, 63]$ (9 biți);
    \item \textbf{0.19\%} au $D \in [-256, 255]$ (12 biți);
    \item Restul utilizează 16 sau 36 de biți.
\end{itemize}

Media ponderată rezultă în aproximativ \textbf{1.5 biți per timestamp}, comparativ cu 64 de biți fără compresie.

\section{Compresia Valorilor: XOR Encoding}
\label{sec:xor}

\subsection{Provocarea compresiei floating-point}

Numerele în virgulă mobilă (floating-point) sunt reprezentate conform standardului IEEE 754. Un număr \textit{double} (64 biți) are structura:

\begin{figure}[h]
\centering
\begin{tikzpicture}[scale=0.8]
    % Sign bit
    \draw (0,0) rectangle (0.5,0.8);
    \node at (0.25,0.4) {S};
    \node at (0.25,-0.3) {\tiny 1 bit};

    % Exponent
    \draw (0.5,0) rectangle (3.5,0.8);
    \node at (2,0.4) {Exponent};
    \node at (2,-0.3) {\tiny 11 biți};

    % Mantissa
    \draw (3.5,0) rectangle (10,0.8);
    \node at (6.75,0.4) {Mantisă (fracție)};
    \node at (6.75,-0.3) {\tiny 52 biți};

    % Bit positions
    \node at (0.25,1.1) {\tiny 63};
    \node at (0.75,1.1) {\tiny 62};
    \node at (3.25,1.1) {\tiny 52};
    \node at (3.75,1.1) {\tiny 51};
    \node at (9.75,1.1) {\tiny 0};
\end{tikzpicture}
\caption{Structura IEEE 754 double-precision (64 biți)}
\label{fig:ieee754}
\end{figure}

Valoarea reprezentată este:
\begin{equation}
    v = (-1)^S \times 2^{E-1023} \times (1 + M/2^{52})
\end{equation}
unde $S$ este bitul de semn, $E$ este exponentul (11 biți), și $M$ este mantisa (52 biți).

\textbf{Problema}: Chiar și pentru valori foarte apropiate, reprezentările binare pot diferi semnificativ la nivelul mantisei. Delta encoding simplu nu funcționează eficient.

\subsection{Observația cheie: XOR pentru valori similare}

Gorilla exploatează o proprietate importantă: dacă două valori floating-point sunt \textit{apropiate numeric}, reprezentările lor binare tind să aibă mulți biți identici, în special în exponent și în biții superiori ai mantisei.

Operația \textbf{XOR} (exclusive OR) între două reprezentări binare produce:
\begin{itemize}
    \item \texttt{0} pentru biții identici;
    \item \texttt{1} pentru biții diferiți.
\end{itemize}

Pentru valori apropiate, XOR-ul va avea mulți \texttt{0} consecutivi la început (\textit{leading zeros}) și la sfârșit (\textit{trailing zeros}).

\begin{example}
Fie $v_1 = 24.0$ și $v_2 = 25.0$:
\begin{align*}
    v_1 &= \texttt{0x4038000000000000} \\
    v_2 &= \texttt{0x4039000000000000} \\
    v_1 \oplus v_2 &= \texttt{0x0001000000000000}
\end{align*}

XOR-ul are 15 leading zeros și 48 trailing zeros, lăsând doar 1 bit semnificativ!
\end{example}

\subsection{Schema de codare XOR}

Gorilla utilizează următoarea schemă pentru codarea valorilor:

\subsubsection{Prima valoare}
Prima valoare din bloc se stochează integral pe 64 de biți (necomprimată).

\subsubsection{Valori ulterioare}
Pentru fiecare valoare $v_n$ după prima:

\begin{enumerate}
    \item Calculăm $X = v_n \oplus v_{n-1}$ (XOR cu valoarea anterioară);

    \item \textbf{Dacă $X = 0$} (valori identice):
    \begin{itemize}
        \item Scriem un singur bit \texttt{0};
    \end{itemize}

    \item \textbf{Dacă $X \neq 0$}:
    \begin{itemize}
        \item Scriem bitul de control \texttt{1};
        \item Calculăm \textit{leading zeros} ($L$) și \textit{trailing zeros} ($T$);
        \item Numărul de biți semnificativi este: $M = 64 - L - T$;
    \end{itemize}

    \item \textbf{Dacă putem refolosi fereastra anterioară} ($L \geq L_{prev}$ și $T \geq T_{prev}$):
    \begin{itemize}
        \item Scriem bitul de control \texttt{0};
        \item Scriem doar biții semnificativi ($M_{prev}$ biți);
    \end{itemize}

    \item \textbf{Altfel} (definim o fereastră nouă):
    \begin{itemize}
        \item Scriem bitul de control \texttt{1};
        \item Scriem $L$ pe 5 biți (permite valori 0-31);
        \item Scriem $M - 1$ pe 6 biți (permite valori 1-64 codate ca 0-63);
        \item Scriem cei $M$ biți semnificativi din XOR.
    \end{itemize}
\end{enumerate}

\begin{table}[h]
\centering
\begin{tabular}{@{}lll@{}}
\toprule
\textbf{Caz} & \textbf{Format} & \textbf{Biți} \\
\midrule
XOR = 0 & \texttt{0} & 1 \\
Refolosim fereastra & \texttt{10} + meaningful bits & 2 + $M_{prev}$ \\
Fereastră nouă & \texttt{11} + 5 biți + 6 biți + meaningful bits & 13 + $M$ \\
\bottomrule
\end{tabular}
\caption{Schema de codare XOR pentru valori}
\label{tab:xor-encoding}
\end{table}

\subsection{Calculul leading și trailing zeros}

Fie $X$ rezultatul XOR reprezentat pe 64 de biți. Definim:

\begin{equation}
    L = \max\{i : \text{biții } 63, 62, \ldots, 64-i \text{ sunt toți } 0\}
\end{equation}

\begin{equation}
    T = \max\{j : \text{biții } 0, 1, \ldots, j-1 \text{ sunt toți } 0\}
\end{equation}

\begin{equation}
    M = 64 - L - T \quad \text{(biții semnificativi)}
\end{equation}

\begin{example}
Pentru $X = \texttt{0x0001000000000000}$:
\begin{itemize}
    \item Reprezentare binară: \texttt{0000...0001 0000...0000} (15 zeros, apoi 1, apoi 48 zeros)
    \item $L = 15$ (leading zeros)
    \item $T = 48$ (trailing zeros)
    \item $M = 64 - 15 - 48 = 1$ (un singur bit semnificativ)
\end{itemize}
\end{example}

\subsection{Algoritmul de codare pentru valori}

\begin{algorithm}[h]
\caption{Codare valoare cu XOR}
\label{alg:val-encode}
\begin{algorithmic}[1]
\Require Valoare $v_n$, valoare anterioară $v_{n-1}$ (ca biți), $L_{prev}$, $T_{prev}$
\Ensure Biți scriși în fluxul de ieșire

\State $X \gets v_n \oplus v_{n-1}$ \Comment{XOR între reprezentări pe 64 biți}

\If{$X = 0$}
    \State \textbf{write} \texttt{0} \Comment{1 bit --- valori identice}
\Else
    \State \textbf{write} \texttt{1} \Comment{bit de control}
    \State $L \gets$ \Call{CountLeadingZeros}{$X$}
    \State $T \gets$ \Call{CountTrailingZeros}{$X$}
    \State $M \gets 64 - L - T$

    \If{$L \geq L_{prev}$ \textbf{and} $T \geq T_{prev}$}
        \State \textbf{write} \texttt{0} \Comment{refolosim fereastra}
        \State $M_{use} \gets 64 - L_{prev} - T_{prev}$
        \State \textbf{write} biții semnificativi din $X$ ($M_{use}$ biți)
    \Else
        \State \textbf{write} \texttt{1} \Comment{fereastră nouă}
        \State \textbf{write} $\min(L, 31)$ pe 5 biți
        \State \textbf{write} $(M - 1)$ pe 6 biți
        \State \textbf{write} biții semnificativi din $X$ ($M$ biți)
        \State $L_{prev} \gets L$; $T_{prev} \gets T$
    \EndIf
\EndIf
\end{algorithmic}
\end{algorithm}

\subsection{Eficiența compresiei valorilor}

În datele Facebook:
\begin{itemize}
    \item \textbf{59.06\%} dintre valori sunt identice cu precedenta (1 bit);
    \item \textbf{28.30\%} refolosesc fereastra anterioară (~27 biți în medie);
    \item \textbf{12.64\%} necesită fereastră nouă (~40 biți în medie).
\end{itemize}

Media rezultă în aproximativ \textbf{12-15 biți per valoare}, comparativ cu 64 de biți fără compresie.

\section{Compresia combinată}

Rata totală de compresie pentru o pereche (timestamp, valoare) este:
\begin{equation}
    R = \frac{128 \text{ biți}}{b_{ts} + b_{val}}
\end{equation}
unde $b_{ts}$ și $b_{val}$ sunt biții necesari pentru timestamp și valoare.

În practică, Gorilla obține în medie \textbf{1.37 bytes per punct} (11 biți), comparativ cu 16 bytes necomprimat, reprezentând o rată de compresie de aproximativ \textbf{12x}.

\section{Analiza complexității}

\subsection{Complexitate temporală}

\textbf{Codare (per punct)}: $O(1)$
\begin{itemize}
    \item Calcul delta/XOR: $O(1)$
    \item Scriere biți: $O(k)$ unde $k$ este numărul de biți (constant, maxim 100)
\end{itemize}

\textbf{Decodare (per punct)}: $O(1)$
\begin{itemize}
    \item Citire biți de control: $O(1)$
    \item Reconstrucție valoare: $O(1)$
\end{itemize}

\subsection{Complexitate spațială}

\textbf{Stare per encoder}: $O(1)$
\begin{itemize}
    \item Timestamp anterior: 8 bytes
    \item Delta anterior: 8 bytes
    \item Valoare anterioară: 8 bytes
    \item Leading/trailing zeros anteriori: 2 bytes
\end{itemize}

\textbf{Buffer de ieșire}: $O(n)$ unde $n$ este numărul de puncte, dar cu factor constant mic ($\approx 1.37$ bytes/punct în loc de 16).

\section{Extindere pentru serii multivariate}

O serie \textbf{multivariată} conține mai multe valori per timestamp:
\begin{equation}
    (t_i, v_i^{(1)}, v_i^{(2)}, \ldots, v_i^{(k)})
\end{equation}

\subsection{Strategia de compresie}

Abordarea naivă ar fi să tratăm fiecare variabilă ca o serie separată, dar aceasta ar duplica timestamp-urile de $k$ ori.

Gorilla pentru serii multivariate utilizează:
\begin{enumerate}
    \item \textbf{Un singur stream de timestamp-uri} (delta-of-delta);
    \item \textbf{Câte un stream separat pentru fiecare variabilă} (XOR encoding).
\end{enumerate}

Toate stream-urile scriu în același buffer de biți, dar fiecare variabilă își menține propriul context XOR (valoare anterioară, leading/trailing zeros).
\chapter{Implementare}

Acest capitol prezintă arhitectura și clasele implementării algoritmului Gorilla în Python. Detaliile teoretice ale algoritmului (schema de codare delta-of-delta, XOR encoding, reprezentarea în complement față de 2) au fost prezentate în Capitolul~\ref{chap:gorilla}. Aici ne concentrăm pe structurile de date, relațiile dintre componente și pe o variantă îmbunătățită a algoritmului original.

\section{Arhitectura sistemului}

\subsection{Diagrama dependențelor}

Sistemul este organizat pe patru niveluri ierarhice, prezentate în Figura~\ref{fig:class-deps}. La bază se află operațiile pe biți, urmate de encoderele și decoderele specializate, iar la vârf se află modulele de stocare.

\begin{figure}[h]
\centering
\begin{tikzpicture}[
    module/.style={draw, rectangle, rounded corners, minimum width=2.8cm, minimum height=0.8cm, align=center, fill=blue!10, font=\small},
    arrow/.style={->, thick, >=stealth}
]
    % Nivelul 1: Operații pe biți (jos)
    \node[module] (bw) at (0,0) {BitWriter};
    \node[module] (br) at (5,0) {BitReader};

    % Nivelul 2: Encodere/Decodere
    \node[module] (tse) at (-1.5,2) {TimestampEncoder};
    \node[module] (ve) at (1.5,2) {ValueEncoder};
    \node[module] (tsd) at (5,2) {TimestampDecoder};
    \node[module] (vd) at (8,2) {ValueDecoder};

    % Nivelul 3: Encoder/Decoder multivariat
    \node[module] (mve) at (0,4) {MultiVariateEncoder};
    \node[module] (mvd) at (6.5,4) {MultiVariateDecoder};

    % Nivelul 4: Block
    \node[module] (mvb) at (3.25,5.5) {MultiVariateBlock};

    % Nivelul 5: Series
    \node[module] (mvs) at (3.25,7) {MultiVariateSeries};

    % Nivelul 6: Store
    \node[module] (store) at (3.25,8.5) {MultiVariateStore};

    % Dependențe Nivel 2 -> Nivel 1
    \draw[arrow] (tse) -- (bw);
    \draw[arrow] (ve) -- (bw);
    \draw[arrow] (tsd) -- (br);
    \draw[arrow] (vd) -- (br);

    % Dependențe Nivel 3 -> Nivel 2
    \draw[arrow] (mve) -- (tse);
    \draw[arrow] (mve) -- (ve);
    \draw[arrow] (mvd) -- (tsd);
    \draw[arrow] (mvd) -- (vd);

    % Dependențe între module stocare
    \draw[arrow] (mvb) -- (mve);
    \draw[arrow] (mvb) -- (mvd);
    \draw[arrow] (mvs) -- (mvb);
    \draw[arrow] (store) -- (mvs);

    % Etichete niveluri
    \node[gray, anchor=east] at (-3,0) {\footnotesize Nivel 1: Biți};
    \node[gray, anchor=east] at (-3,2) {\footnotesize Nivel 2: Codare};
    \node[gray, anchor=east] at (-3,4) {\footnotesize Nivel 3: Agregare};
    \node[gray, anchor=east] at (-3,6.75) {\footnotesize Nivel 4: Stocare};
\end{tikzpicture}
\caption{Ierarhia dependențelor dintre clase}
\label{fig:class-deps}
\end{figure}

\subsection{Principii de design}

Toate clasele utilizează directiva \texttt{\_\_slots\_\_} pentru optimizare. Aceasta elimină dicționarul implicit al obiectelor Python (\texttt{\_\_dict\_\_}), rezultând în:
\begin{itemize}
    \item Reducerea amprentei de memorie per obiect
    \item Acces mai rapid la atribute (fără lookup în dicționar)
    \item Protecție la erori de tip (atributele nedeclarate generează excepții)
\end{itemize}

\section{Nivelul 1: Operații pe biți}

\subsection{Clasa BitWriter}

\textbf{Rol}: Permite scrierea de secvențe de biți de lungime arbitrară într-un buffer de bytes. Algoritmul Gorilla necesită scrierea de câmpuri de 1, 5, 7, 9, 12 sau 32 de biți care nu se aliniază la granița de byte.

\textbf{Structura de date internă}:
\begin{center}
\begin{tabular}{|l|l|p{7cm}|}
\hline
\textbf{Atribut} & \textbf{Tip} & \textbf{Descriere} \\
\hline
\texttt{\_buf} & \texttt{bytearray} & Buffer cu bytes compleți (finalizați) \\
\texttt{\_cur} & \texttt{int} & Byte parțial în construcție (0--255) \\
\texttt{\_nbits} & \texttt{int} & Numărul de biți scriși în \texttt{\_cur} (0--7) \\
\hline
\end{tabular}
\end{center}

\textbf{Modelul de acumulare}: Biții se scriu de la stânga la dreapta (MSB-first). Când \texttt{\_nbits} atinge 8, byte-ul complet este mutat în buffer și se resetează \texttt{\_cur}.

\begin{center}
\begin{tabular}{|l|p{8cm}|}
\hline
\textbf{Metodă} & \textbf{Descriere} \\
\hline
\texttt{write\_bit(bit)} & Scrie un singur bit (0 sau 1) \\
\texttt{write\_bits(x, n)} & Scrie $n$ biți din valoarea $x$ \\
\texttt{write\_signed(x, bits)} & Scrie $x$ ca număr cu semn (complement față de 2) \\
\texttt{write\_u64(x)} / \texttt{write\_i64(x)} & Scrie întreg pe 64 biți (aliniat la byte) \\
\texttt{to\_bytes()} & Finalizează și returnează \texttt{bytes} \\
\hline
\end{tabular}
\end{center}

\subsection{Clasa BitReader}

\textbf{Rol}: Clasă complementară pentru citirea bit cu bit dintr-un flux de bytes. Reconstruiește valorile scrise cu BitWriter.

\textbf{Structura de date internă}:
\begin{center}
\begin{tabular}{|l|l|p{7cm}|}
\hline
\textbf{Atribut} & \textbf{Tip} & \textbf{Descriere} \\
\hline
\texttt{\_data} & \texttt{bytes} & Sursa de date (imutabilă) \\
\texttt{\_byte\_pos} & \texttt{int} & Indexul byte-ului curent \\
\texttt{\_bit\_pos} & \texttt{int} & Poziția bitului în byte (0--7) \\
\hline
\end{tabular}
\end{center}

\begin{center}
\begin{tabular}{|l|p{8cm}|}
\hline
\textbf{Metodă} & \textbf{Descriere} \\
\hline
\texttt{read\_bit()} & Citește un singur bit \\
\texttt{read\_bits(n)} & Citește $n$ biți ca întreg unsigned \\
\texttt{read\_signed(bits)} & Citește număr cu semn (complement față de 2) \\
\texttt{read\_u64()} / \texttt{read\_i64()} & Citește întreg pe 64 biți \\
\texttt{bits\_remaining} & Proprietate: biți rămași în flux \\
\hline
\end{tabular}
\end{center}

\section{Nivelul 2: Encodere și decodere}

Schema de codare delta-of-delta pentru timestamp-uri și XOR encoding pentru valori au fost detaliate în Capitolul 3. Acum prezentăm structura internă a claselor care implementează aceste scheme.

\subsection{Clasa TimestampEncoder}

\textbf{Rol}: Comprimă timestamp-uri folosind codarea delta-of-delta.

\begin{center}
\begin{tabular}{|l|p{9cm}|}
\hline
\textbf{Atribut} & \textbf{Semnificație} \\
\hline
\texttt{\_writer} & Referință la BitWriter pentru scriere \\
\texttt{\_prev\_timestamp} & Ultimul timestamp procesat ($t_{i-1}$) \\
\texttt{\_prev\_delta} & Ultimul delta calculat ($\delta_{i-1}$) \\
\texttt{\_count} & Numărul de timestamp-uri adăugate \\
\hline
\end{tabular}
\end{center}

\textbf{Logica de codare}:
\begin{itemize}
    \item Primul timestamp: se scrie complet pe 64 de biți
    \item Al doilea timestamp: se scrie $\delta_1 = t_1 - t_0$ pe 64 de biți
    \item Timestamp-urile următoare: se codează delta-of-delta conform schemei din Tabelul~\ref{tab:dod-encoding}
\end{itemize}

\subsection{Clasa TimestampDecoder}

\textbf{Rol}: Reconstruiește timestamp-urile din fluxul comprimat. Inversează operațiile TimestampEncoder.

\textbf{Starea internă}: Identică cu TimestampEncoder, dar folosește BitReader în loc de BitWriter.

\subsection{Clasa ValueEncoder}

\textbf{Rol}: Comprimă valori \texttt{float64} folosind codarea XOR.

\begin{center}
\begin{tabular}{|l|p{8cm}|}
\hline
\textbf{Atribut} & \textbf{Semnificație} \\
\hline
\texttt{\_writer} & Referință la BitWriter \\
\texttt{\_prev\_value\_bits} & Reprezentarea pe biți a valorii anterioare \\
\texttt{\_prev\_leading} & Numărul de zerouri din față (fereastra anterioară) \\
\texttt{\_prev\_trailing} & Numărul de zerouri din coadă (fereastra anterioară) \\
\texttt{\_count} & Numărul de valori procesate \\
\hline
\end{tabular}
\end{center}


\subsection{Clasa ValueDecoder}

\textbf{Rol}: Reconstruiește valorile float din fluxul comprimat XOR.

\section{Nivelul 3: Stocare multivariată}

\subsection{Clasa MultiVariateBlock}

\textbf{Rol}: Gestionează un bloc de date pentru serii cu multiple variabile. Toate variabilele împart același stream de timestamp-uri, evitând duplicarea.

\textbf{Structura de date}:
\begin{center}
\begin{tabular}{|l|p{8cm}|}
\hline
\textbf{Atribut} & \textbf{Descriere} \\
\hline
\texttt{\_writer} & BitWriter comun pentru toate stream-urile \\
\texttt{\_ts\_encoder} & Un singur TimestampEncoder \\
\texttt{\_val\_encoders} & Dicționar: nume variabilă $\rightarrow$ ValueEncoder \\
\texttt{\_var\_names} & Lista ordonată a numelor variabilelor \\
\texttt{\_count} & Numărul de puncte din bloc \\
\texttt{\_closed} & Flag: blocul este finalizat \\
\texttt{\_start\_timestamp} & Timestamp-ul de start al blocului \\
\hline
\end{tabular}
\end{center}

\textbf{Principiul cheie}: Un singur timestamp pentru $N$ variabile. Ordinea variabilelor trebuie să fie \textbf{deterministă} la codare și decodare.

\begin{center}
\begin{tabular}{|l|p{8cm}|}
\hline
\textbf{Metodă} & \textbf{Descriere} \\
\hline
\texttt{add(ts, values)} & Adaugă punct (algoritmul Gorilla standard) \\
\texttt{add\_verification(ts, values)} & Adaugă punct (varianta îmbunătățită) \\
\texttt{seal()} & Finalizează blocul, returnează bytes comprimați \\
\hline
\end{tabular}
\end{center}

\subsection{Clasa MultiVariateDecoder}

\textbf{Rol}: Citește și reconstruiește punctele multivariate dintr-un bloc comprimat.

\textbf{Structură}: Un TimestampDecoder + un dicționar de ValueDecodere (unul per variabilă).

\subsection{Clasa MultiVariateSeries}

\textbf{Rol}: Gestionează o serie temporală completă, organizată în blocuri de durată fixă (implicit 2 ore).

\textbf{Structura de date}:
\begin{center}
\begin{tabular}{|l|p{8cm}|}
\hline
\textbf{Atribut} & \textbf{Descriere} \\
\hline
\texttt{\_var\_names} & Lista numelor variabilelor \\
\texttt{\_block\_duration} & Durata unui bloc în milisecunde \\
\texttt{\_open\_block} & Blocul curent (în care se scrie) \\
\texttt{\_closed\_blocks} & Listă: (start\_ts, count, bytes) \\
\hline
\end{tabular}
\end{center}

\textbf{Interfață}:
\begin{center}
\begin{tabular}{|l|p{8cm}|}
\hline
\textbf{Metodă} & \textbf{Descriere} \\
\hline
\texttt{insert(ts, values)} & Inserează punct (gestionează blocuri automat) \\
\texttt{flush()} & Închide blocul curent \\
\texttt{query(t\_start, t\_end)} & Interogare pe interval temporal \\
\texttt{get\_compression\_stats()} & Returnează statistici de compresie \\
\hline
\end{tabular}
\end{center}

\subsection{Clasa MultiVariateStore}

\textbf{Rol}: Container pentru mai multe serii temporale multivariate, identificate prin chei unice.

\textbf{Structură}: Dicționar \texttt{serie\_key} $\rightarrow$ \texttt{MultiVariateSeries}.

\section{Varianta îmbunătățită: Optimizarea ferestrei XOR}

În această secțiune prezentăm o modificare a algoritmului Gorilla original, implementată în metodele \texttt{add\_value\_verification} și \texttt{add\_verification}.

\subsection{Problema algoritmului original}

În algoritmul Gorilla standard, decizia de a refolosi fereastra anterioară se bazează exclusiv pe condiția:
\begin{equation}
    L_{curent} \geq L_{anterior} \quad \text{și} \quad T_{curent} \geq T_{anterior}
\end{equation}

unde $L$ = leading zeros și $T$ = trailing zeros.

\textbf{Problema}: Această condiție poate duce la ineficiențe. Dacă fereastra anterioară are $M_{anterior} = 64 - L_{anterior} - T_{anterior}$ biți semnificativi, iar valoarea curentă ar necesita doar $M_{curent} = 64 - L_{curent} - T_{curent}$ biți, algoritmul original va scrie $M_{anterior}$ biți chiar dacă $M_{curent} \ll M_{anterior}$.

\begin{example}
Fie fereastra anterioară cu $L_{ant} = 10$, $T_{ant} = 20$, deci $M_{ant} = 34$ biți.

Valoarea curentă are $L_{cur} = 15$, $T_{cur} = 40$, deci $M_{cur} = 9$ biți.

Condiția originală ($15 \geq 10$ și $40 \geq 20$) este satisfăcută, deci algoritmul va refolosi fereastra și va scrie 34 de biți în loc de 9.

\textbf{Costul refolosirii}: 2 (prefix \texttt{10}) + 34 (biți) = 36 biți

\textbf{Costul ferestrei noi}: 2 (prefix \texttt{11}) + 5 (leading) + 6 (lungime) + 9 (biți) = 22 biți

Diferența: 14 biți irosiți per valoare!
\end{example}

\subsection{Soluția propusă}

Varianta îmbunătățită adaugă o condiție suplimentară: se verifică dacă fereastra anterioară este \textbf{semnificativ} mai mare decât fereastra curentă. Crearea unei ferestre noi costă exact 11 biți (5 pentru leading zeros + 6 pentru lungime). Prin urmare, merită să creăm o fereastră nouă doar dacă economia de biți depășește acest overhead:

\begin{equation}
    M_{anterior} - M_{curent} > 11
\end{equation}

\textbf{Condiția completă pentru refolosirea ferestrei}:
\begin{equation}
    L_{cur} \geq L_{ant} \quad \land \quad T_{cur} \geq T_{ant} \quad \land \quad (M_{ant} - M_{cur} \leq 11)
\end{equation}

\subsection{Implementarea}

În fișierul \texttt{value\_compression.py}, metoda \texttt{add\_value\_verification} implementează această logică:

\begin{lstlisting}[language=Python, caption={Condiția îmbunătățită pentru refolosirea ferestrei}]
# Varianta standard (add_value)
if (self._prev_leading != 255 and
    leading >= self._prev_leading and
    trailing >= self._prev_trailing):
    # Refolosim fereastra

# Varianta imbunatatita (add_value_verification)
if (self._prev_leading != 255 and
    leading >= self._prev_leading and
    trailing >= self._prev_trailing and
    not ((64 - self._prev_trailing - self._prev_leading)
         - (64 - trailing - leading) > 11)):
    # Refolosim fereastra doar daca diferenta <= 11 biti
\end{lstlisting}

Condiția \texttt{not (M\_ant - M\_cur > 11)} se traduce în: ``refolosește fereastra doar dacă economia potențială nu depășește overhead-ul de 11 biți''.

\subsection{Analiza trade-off-ului}

Decizia optimă se bazează pe compararea costurilor:

\begin{center}
\begin{tabular}{|l|c|}
\hline
\textbf{Opțiune} & \textbf{Cost (biți)} \\
\hline
Refolosire fereastră & $2 + M_{anterior}$ \\
Fereastră nouă & $2 + 11 + M_{curent}$ \\
\hline
\end{tabular}
\end{center}

Fereastra nouă este mai eficientă când:
\begin{equation}
    2 + 11 + M_{curent} < 2 + M_{anterior} \implies M_{anterior} - M_{curent} > 11
\end{equation}

Această condiție este \textbf{matematic optimă}: se creează fereastră nouă exact atunci când este benefic.

\section{Fluxul de date}

\subsection{Codare (compresie)}

\begin{enumerate}
    \item Aplicația apelează \texttt{MultiVariateSeries.insert(ts, values)}
    \item Seria verifică dacă e nevoie de bloc nou și apelează \texttt{MultiVariateBlock.add()}
    \item Blocul apelează \texttt{TimestampEncoder.add\_timestamp(ts)}
    \item Encoder-ul timestamp calculează delta-of-delta și scrie în BitWriter
    \item Pentru fiecare variabilă, blocul apelează \texttt{ValueEncoder.add\_value(v)}
    \item Fiecare ValueEncoder calculează XOR și scrie în același BitWriter
    \item La \texttt{flush()}, BitWriter returnează bytes-ii comprimați
\end{enumerate}

\subsection{Decodare (decompresie)}

\begin{enumerate}
    \item Se creează \texttt{MultiVariateDecoder} cu bytes comprimați
    \item Decoder-ul inițializează BitReader + TimestampDecoder + ValueDecoders
    \item La fiecare \texttt{read\_point()}:
    \begin{itemize}
        \item TimestampDecoder citește și reconstruiește timestamp-ul
        \item Fiecare ValueDecoder citește și reconstruiește valoarea sa
    \end{itemize}
    \item Se returnează tuplu (timestamp, dicționar\_valori)
\end{enumerate}
\chapter{Experimente și Rezultate}

Acest capitol prezintă evaluarea experimentală a implementării algoritmului Gorilla. Sunt descrise seturile de date utilizate, metodologia de testare și rezultatele obținute.

\section{Seturi de date}

Pentru validarea implementării au fost utilizate două seturi de date cu caracteristici diferite:

\subsection{CPU Load Average (serie univariată)}

Primul set de date conține metrici de încărcare CPU colectate de pe un sistem Linux.

\begin{table}[h]
\centering
\begin{tabular}{@{}ll@{}}
\toprule
\textbf{Caracteristică} & \textbf{Valoare} \\
\midrule
Număr de puncte & 480 \\
Perioadă de colectare & $\sim$16 ore \\
Interval mediu & $\sim$120 secunde \\
Tip date & Univariat (load average) \\
Interval valori & 0.57 -- 2.55 \\
\bottomrule
\end{tabular}
\caption{Caracteristicile setului de date CPU Load}
\label{tab:cpu-dataset}
\end{table}

\textbf{Load average} reprezintă numărul mediu de procese în coadă de execuție pe o fereastră de timp. Valorile variază ușor în timp, cu schimbări graduale caracteristice sarcinilor tipice de server.

\subsection{Room Climate Dataset (serie multivariată)}

Al doilea set de date provine din Room Climate Dataset, o colecție de măsurători de la senzori de mediu interior \cite{roomclimate}.

\begin{table}[h]
\centering
\begin{tabular}{@{}ll@{}}
\toprule
\textbf{Caracteristică} & \textbf{Valoare} \\
\midrule
Număr de puncte & 68,229 \\
Perioadă de colectare & $\sim$22 zile \\
Interval mediu & $\sim$28 secunde \\
Număr variabile & 8 \\
Node ID filtrat & NID = 1 \\
\bottomrule
\end{tabular}
\caption{Caracteristicile setului de date Room Climate}
\label{tab:room-dataset}
\end{table}

Cele 8 variabile măsurate sunt:

\begin{enumerate}
    \item \textbf{Temperatură} (°C): 17--26°C
    \item \textbf{Umiditate relativă} (\%): 25--60\%
    \item \textbf{Lumină 1} (lux): senzor principal de lumină
    \item \textbf{Lumină 2} (lux): senzor secundar
    \item \textbf{Ocupare}: indicator binar (0/1)
    \item \textbf{Activitate}: nivel de mișcare detectată
    \item \textbf{Ușă}: stare ușă (deschis/închis)
    \item \textbf{Fereastră}: stare fereastră (deschis/închis)
\end{enumerate}

\section{Metodologia experimentală}

\subsection{Metrici de evaluare}

Au fost calculate următoarele metrici:

\begin{enumerate}
    \item \textbf{Dimensiune originală}: $n \times (8 + 8k)$ bytes, unde $n$ este numărul de puncte și $k$ numărul de variabile (1 pentru univariat).

    \item \textbf{Dimensiune comprimată}: numărul de bytes după compresie.

    \item \textbf{Rata de compresie}:
    \begin{equation}
        R = \frac{\text{Dimensiune originală}}{\text{Dimensiune comprimată}}
    \end{equation}

    \item \textbf{Bytes per punct}:
    \begin{equation}
        B = \frac{\text{Dimensiune comprimată}}{n}
    \end{equation}

    \item \textbf{Economie procentuală}:
    \begin{equation}
        E = \left(1 - \frac{1}{R}\right) \times 100\%
    \end{equation}
\end{enumerate}

\subsection{Configurația blocurilor}

Testele au fost realizate cu blocuri de 2 ore (7.200.000 ms), conform configurației standard Gorilla.

\subsection{Validare round-trip}

Pentru a verifica corectitudinea implementării, fiecare test include o validare \textit{round-trip}: datele sunt comprimate, apoi decomprimate, iar valorile rezultate sunt comparate bit-cu-bit cu originalele.

\section{Rezultate: Serie univariată (CPU Load)}

\subsection{Statistici de compresie}

\begin{table}[h]
\centering
\begin{tabular}{@{}lr@{}}
\toprule
\textbf{Metrică} & \textbf{Valoare} \\
\midrule
Puncte totale & 480 \\
Blocuri (2 ore) & 5 \\
Dimensiune originală & 7,680 bytes \\
Dimensiune comprimată & 3,412 bytes \\
Rata de compresie & 2.25x \\
Bytes per punct & 7.11 \\
Economie & 55.6\% \\
\bottomrule
\end{tabular}
\caption{Rezultate compresie pentru CPU Load}
\label{tab:cpu-results}
\end{table}

\subsection{Analiza rezultatelor}

Rata de compresie de 2.25x este mai modestă decât cea raportată în articolul Gorilla ($\sim$12x) din următoarele motive:

\begin{enumerate}
    \item \textbf{Interval neregulat}: Timestamp-urile din acest set nu sunt perfect periodice (interval mediu de $\sim$120s cu variații), ceea ce reduce eficiența delta-of-delta.

    \item \textbf{Valori variabile}: Load average variază frecvent, ceea ce produce mai multe XOR-uri non-zero.

    \item \textbf{Set de date mic}: Cu doar 480 de puncte, overhead-ul primelor valori (scrise integral) are impact proporțional mai mare.
\end{enumerate}

\section{Rezultate: Serie multivariată (Room Climate)}

\subsection{Statistici de compresie}

\begin{table}[h]
\centering
\begin{tabular}{@{}lr@{}}
\toprule
\textbf{Metrică} & \textbf{Valoare} \\
\midrule
Puncte totale & 68,229 \\
Variabile & 8 \\
Blocuri (2 ore) & 262 \\
Dimensiune originală & 4,912,488 bytes \\
Dimensiune comprimată & 1,190,271 bytes \\
Rata de compresie & 4.13x \\
Bytes per punct & 17.45 \\
Economie & 75.8\% \\
\bottomrule
\end{tabular}
\caption{Rezultate compresie pentru Room Climate}
\label{tab:room-results}
\end{table}

\subsection{Analiza pe variabile}

Diferitele variabile din setul Room Climate au comportamente distincte care afectează compresia:

\begin{itemize}
    \item \textbf{Temperatură și umiditate}: Variază lent și continuu, producând multe XOR-uri cu puțini biți semnificativi. Compresie foarte bună.

    \item \textbf{Variabile binare} (ocupare, ușă, fereastră): Valorile sunt 0 sau 1, producând fie XOR = 0 (identic), fie diferențe mari când se schimbă starea.

    \item \textbf{Senzori de lumină}: Variabilitate moderată, dar cu tendința de a rămâne constanți pe perioade lungi (noapte vs. zi).
\end{itemize}

\subsection{Comparație cu stocarea separată}

Dacă am fi tratat fiecare variabilă ca o serie separată (duplicând timestamp-urile), dimensiunea ar fi fost:

\begin{equation}
    \text{Dimensiune separată} = 8 \times 68,229 \times 16 = 8,733,312 \text{ bytes}
\end{equation}

Stocarea multivariată (un singur stream de timestamp-uri) economisește:
\begin{equation}
    8,733,312 - 4,912,488 = 3,820,824 \text{ bytes} \approx 3.6 \text{ MB}
\end{equation}

doar din evitarea duplicării timestamp-urilor, înainte de compresie.

\section{Distribuția biților}

\subsection{Timestamp-uri}

Analiza distribuției valorilor delta-of-delta pentru setul Room Climate:

\begin{table}[h]
\centering
\begin{tabular}{@{}lrr@{}}
\toprule
\textbf{Bucket} & \textbf{Procent} & \textbf{Biți} \\
\midrule
$D = 0$ & $\sim$85\% & 1 \\
$D \in [-64, 63]$ & $\sim$12\% & 9 \\
$D \in [-256, 255]$ & $\sim$2\% & 12 \\
$D \in [-2048, 2047]$ & $\sim$0.8\% & 16 \\
Altfel & $\sim$0.2\% & 36 \\
\bottomrule
\end{tabular}
\caption{Distribuția delta-of-delta pentru Room Climate}
\label{tab:dod-dist}
\end{table}

Media ponderată: $\sim$2.3 biți per timestamp.

\subsection{Valori}

Distribuția tipurilor de codare XOR:

\begin{table}[h]
\centering
\begin{tabular}{@{}lrr@{}}
\toprule
\textbf{Caz} & \textbf{Procent} & \textbf{Biți medii} \\
\midrule
XOR = 0 (identic) & $\sim$45\% & 1 \\
Refolosire fereastră & $\sim$35\% & $\sim$25 \\
Fereastră nouă & $\sim$20\% & $\sim$40 \\
\bottomrule
\end{tabular}
\caption{Distribuția codării XOR pentru Room Climate}
\label{tab:xor-dist}
\end{table}

\section{Efectul dimensiunii blocului}

S-a testat impactul dimensiunii blocului asupra ratei de compresie:

\begin{table}[h]
\centering
\begin{tabular}{@{}lcc@{}}
\toprule
\textbf{Dimensiune bloc} & \textbf{Rata compresie} & \textbf{Nr. blocuri} \\
\midrule
30 minute & 3.8x & 1048 \\
1 oră & 4.0x & 524 \\
2 ore (standard) & 4.13x & 262 \\
4 ore & 4.2x & 131 \\
\bottomrule
\end{tabular}
\caption{Impactul dimensiunii blocului}
\label{tab:block-size}
\end{table}

Blocurile mai mari oferă compresie mai bună (contextul XOR se păstrează mai mult), dar:
\begin{itemize}
    \item Cresc latența pentru interogări pe intervale scurte;
    \item Necesită mai multă memorie pentru decodare;
    \item Sunt mai sensibile la corupție (pierderea unui bloc = pierdere mai mare de date).
\end{itemize}

Pragul de 2 ore reprezintă un compromis rezonabil, validat empiric de echipa Facebook.

\section{Validarea corectitudinii}

Toate testele au trecut validarea round-trip:

\begin{lstlisting}[language=Python, caption={Verificare round-trip}]
# Comprima
compressed = series.get_compressed_data()

# Decomprima
decoded = decode_series(compressed, count, var_names)

# Verifica
for original, decoded in zip(original_data, decoded_data):
    assert original.timestamp == decoded.timestamp
    for var in var_names:
        assert original.values[var] == decoded.values[var]

print("[OK] Round-trip validation passed!")
\end{lstlisting}

Toate valorile sunt recuperate exact (bit-perfect), confirmând caracterul \textit{lossless} al compresiei.

\section{Comparație cu articolul original}

\begin{table}[h]
\centering
\begin{tabular}{@{}lcc@{}}
\toprule
\textbf{Metrică} & \textbf{Gorilla (Facebook)} & \textbf{Implementarea noastră} \\
\midrule
Bytes/punct (medie) & 1.37 & 4--7 \\
\% timestamps cu D=0 & 96.39\% & 85\% \\
\% valori XOR=0 & 59.06\% & 45\% \\
Rata compresie & 12x & 2--4x \\
\bottomrule
\end{tabular}
\caption{Comparație cu rezultatele raportate în articolul Gorilla}
\label{tab:comparison}
\end{table}

Diferențele se explică prin:

\begin{enumerate}
    \item \textbf{Caracteristicile datelor}: Datele Facebook provin de la sisteme de monitorizare foarte regulate (interval fix de 60s), în timp ce seturile noastre au mai mult jitter.

    \item \textbf{Tipul valorilor}: Monitorizarea Facebook folosește preponderent contoare (integers convertiți la double), care produc XOR-uri mai compresibile decât măsurătorile analogice de temperatură.

    \item \textbf{Scala}: La 2 miliarde de serii, efectele statistice se netezesc; la câteva sute de puncte, variabilitatea este mai mare.
\end{enumerate}

Cu toate acestea, implementarea demonstrează corectitudinea algoritmului și obține rate de compresie semnificative (2--4x) pe date reale diverse.
\chapter{Concluzii}

\section{Sinteza contribuțiilor}

Prezentul proiect a implementat și analizat algoritmul de compresie Gorilla pentru serii temporale. Principalele contribuții sunt:

\subsection{Implementare completă în Python}

A fost realizată o implementare funcțională a algoritmului Gorilla, incluzând:
\begin{itemize}
    \item Module pentru operații la nivel de bit (\texttt{BitWriter}, \texttt{BitReader});
    \item Compresie delta-of-delta pentru timestamp-uri;
    \item Compresie XOR pentru valori în virgulă mobilă;
    \item Suport pentru serii univariate și multivariate;
    \item Organizare în blocuri de durată configurabilă.
\end{itemize}

\subsection{Documentare teoretică}

Au fost prezentate în detaliu:
\begin{itemize}
    \item Fundamentele matematice ale reprezentării în complement față de 2;
    \item Schema de codare variabilă pentru delta-of-delta;
    \item Mecanismul XOR și optimizarea prin refolosirea ferestrei de biți;
    \item Analiza complexității algoritmilor.
\end{itemize}

\subsection{Validare experimentală}

Implementarea a fost testată pe două seturi de date reale:
\begin{itemize}
    \item Serie univariată (CPU Load): rată de compresie 2.25x;
    \item Serie multivariată (Room Climate, 8 variabile): rată de compresie 4.13x.
\end{itemize}

Toate testele au validat corectitudinea round-trip a compresiei (lossless).

\section{Concluzii tehnice}

Pe baza experimentelor realizate, putem formula următoarele concluzii:

\subsection{Eficiența delta-of-delta}

Compresia timestamp-urilor este extrem de eficientă pentru date cu intervale regulate:
\begin{itemize}
    \item Aproximativ 85\% dintre timestamp-uri sunt comprimate la 1 singur bit;
    \item Media este de 2--3 biți per timestamp, față de 64 biți necomprimat;
    \item Eficiența scade semnificativ pentru date cu jitter mare sau intervale aleatorii.
\end{itemize}

\subsection{Eficiența XOR encoding}

Compresia valorilor depinde de caracteristicile datelor:
\begin{itemize}
    \item Valori constante sau aproape constante se comprimă foarte bine (1 bit);
    \item Valori cu variație lentă beneficiază de refolosirea ferestrei;
    \item Valori aleatorii sau cu salturi mari necesită ferestre noi (13+ biți overhead).
\end{itemize}

\subsection{Avantajul stocării multivariate}

Pentru serii cu multiple variabile per timestamp, abordarea cu un singur stream de timestamp-uri este semnificativ mai eficientă:
\begin{itemize}
    \item Evită duplicarea timestamp-urilor ($k-1$ copii eliminate);
    \item Fiecare variabilă menține context XOR propriu;
    \item Economia este proporțională cu numărul de variabile.
\end{itemize}

\section{Limitări identificate}

\subsection{Date cu variabilitate mare}

Algoritmul nu este optim pentru:
\begin{itemize}
    \item Timestamp-uri aleatorii sau foarte neregulate;
    \item Valori care variază drastic între măsurători consecutive;
    \item Date criptate sau comprimate anterior (aspect aleatoriu).
\end{itemize}

\subsection{Overhead pentru serii scurte}

Pentru serii cu foarte puține puncte ($<$100), overhead-ul primelor valori (stocate integral) domină, reducând eficiența compresiei.

\subsection{Decodare secvențială}

Decomprimarea necesită parcurgerea secvențială de la începutul blocului. Nu există acces aleatoriu (random access) la un punct specific fără a decoda toate punctele anterioare din bloc.

\subsection{Implementare interpretată}

Fiind scrisă în Python (limbaj interpretat), implementarea are performanță inferioară unei implementări native în C/C++ sau Rust. Pentru aplicații de producție cu rate mari de ingestie, ar fi necesară o implementare compilată.

\section{Direcții de dezvoltare ulterioară}

\subsection{Optimizări de performanță}

\begin{itemize}
    \item \textbf{Implementare în C/Rust}: Port-area la un limbaj compilat pentru performanță de producție;
    \item \textbf{Utilizarea SIMD}: Operațiile pe biți pot beneficia de instrucțiuni vectoriale;
    \item \textbf{Paralelizare}: Blocurile independente pot fi comprimate/decomprimate în paralel.
\end{itemize}

\subsection{Funcționalități adiționale}

\begin{itemize}
    \item \textbf{Partial decode}: Decodare doar a unui interval din bloc, fără a parcurge tot;
    \item \textbf{Indexare}: Structuri de indexare pentru acces rapid la anumite timestamp-uri;
    \item \textbf{Streaming API}: Interfață pentru procesare în flux (streaming) a datelor;
    \item \textbf{Suport pentru missing values}: Gestionarea explicită a punctelor lipsă.
\end{itemize}

\subsection{Integrare}

\begin{itemize}
    \item \textbf{Format de fișier persistent}: Definirea unui format binar pentru stocare pe disc;
    \item \textbf{Integrare cu baze de date}: Plugin-uri pentru PostgreSQL, ClickHouse etc.;
    \item \textbf{API REST}: Serviciu HTTP pentru ingestie și interogare.
\end{itemize}

\section{Concluzii finale}

Algoritmul Gorilla reprezintă o soluție elegantă și eficientă pentru compresia seriilor temporale. Prin exploatarea inteligentă a proprietăților specifice ale acestui tip de date --- periodicitatea timestamp-urilor și corelația temporală a valorilor --- obține rate de compresie semnificative (2--12x în funcție de date) menținând caracterul lossless.

Implementarea realizată în cadrul acestui proiect demonstrează fezabilitatea aplicării algoritmului pe date reale și oferă o bază pentru dezvoltări ulterioare. Deși Python nu este limbajul optim pentru performanță maximă, claritatea codului și ușurința testării au facilitat înțelegerea profundă a algoritmului.

Compresia seriilor temporale rămâne un domeniu activ de cercetare, cu aplicații în creștere odată cu proliferarea IoT și a sistemelor de monitorizare distribuite. Gorilla, alături de tehnici mai noi precum Gorilla v2 și variantele sale, continuă să fie relevant pentru sistemele care necesită stocare eficientă în memorie cu acces rapid la date recente.

\printbibliography[heading=bibintoc]

\end{document}
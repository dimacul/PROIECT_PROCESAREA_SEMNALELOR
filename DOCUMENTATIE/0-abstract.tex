\begin{abstractpage}

\begin{abstract}{romanian}
Sistemele moderne de monitorizare generează volume imense de date sub formă de serii de timp --- secvențe ordonate 
de perechi \textit{(timestamp, value)} care înregistrează evoluția în timp a diferitelor metrici. Stocarea eficientă 
a acestor date reprezintă o provocare semnificativă, deoarece abordarea naivă necesită 16 bytes pentru fiecare datapoint 
(8 bytes pentru timestamp și 8 pentru valoarea în format IEEE 754 double).

În acest proiect analizăm și implementăm algoritmul de compresie \textbf{Gorilla}, dezvoltat de Facebook pentru baza lor de date de serii de timp in-memory. 
Soluția propusă combină două tehnici complementare: \textit{delta-of-delta encoding} pentru comprimarea timestamp-urilor 
și \textit{XOR encoding} pentru comprimarea valorilor în virgulă mobilă. 
Ambele tehnici exploatează proprietățile specifice ale seriilor temporale --- periodicitatea timestamp-urilor și 
corelația temporală a valorilor --- pentru a obține rate de compresie semnificative.

Implementarea realizată de noi în Python demonstrează eficiența algoritmului pe două seturi de date: metrici de utilizare CPU (serie univariată) și 
date climatice dintr-un mediu interior (serie multivariată cu 8 variabile). 
Rezultatele experimentale arată rate de compresie de aproximativ 2--4x, reducând consumul de memorie de la 16 bytes la aproximativ 4--8 bytes per datapoint, 
în funcție de caracteristicile datelor.
\end{abstract}



\end{abstractpage}
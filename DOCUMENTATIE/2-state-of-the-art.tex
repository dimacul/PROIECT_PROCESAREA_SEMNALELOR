\chapter{State-of-the-Art}

Înainte de a prezenta algoritmul Gorilla, vom trece în revistă provocările cărora bazele de date cu serii de timp încearcă să le facă față și compromisurile pe care le fac.

\section{Baze de date pentru serii de timp (TSDB)}

O \textbf{bază de date pentru serii temporale} (Time Series Database --- TSDB) este un sistem software optimizat pentru stocarea și interogarea datelor indexate după timp. Spre deosebire de bazele de date relaționale tradiționale, TSDB-urile sunt proiectate pentru:

\begin{itemize}
    \item Rate foarte mari de scriere (milioane de puncte pe secundă);
    \item Interogări pe intervale de timp (,,dă-mi toate valorile între ora 10:00 și 11:00'');
    \item Agregări temporale (medii, sume, percentile pe ferestre de timp);
    \item Retenție automată a datelor (ștergerea datelor mai vechi de X zile).
\end{itemize}

\section{OpenTSDB}

\textbf{OpenTSDB} (Open Time Series Database) este o bază de date distribuită pentru serii temporale, construită peste Apache HBase \cite{opentsdb}. HBase, la rândul său, este un sistem de stocare distribuit bazat pe modelul BigTable al Google.

\subsection{Arhitectură}

OpenTSDB utilizează un model de date în care fiecare serie temporală este identificată printr-un set de perechi cheie-valoare numite \textit{tag-uri}:

\begin{verbatim}
    metric: sys.cpu.user
    tags: {host=webserver01, cpu=0}
    timestamp: 1458034800
    value: 42.5
\end{verbatim}

Datele sunt stocate în HBase într-o structură de tabel optimizată pentru scanări secvențiale pe intervale de timp.

\subsection{Avantaje și limitări}

\textbf{Avantaje:}
\begin{itemize}
    \item Scalabilitate orizontală prin adăugarea de noduri HBase;
    \item Persistență durabilă pe disc cu replicare;
    \item Model de date flexibil cu tag-uri arbitrare.
\end{itemize}

\textbf{Limitări:}
\begin{itemize}
    \item \textbf{Latență ridicată}: Fiind bazat pe disc, interogările pot dura secunde sau chiar minute pentru volume mari de date;
    \item \textbf{Complexitate operațională}: Necesită administrarea unui cluster HBase și Hadoop;
    \item \textbf{Nu este optimizat pentru date recente}: Toate datele sunt tratate egal, fără prioritizare a celor recente.
\end{itemize}

În contextul Facebook, OpenTSDB (prin HBase) nu putea scala pentru cerințele de latență necesare. Percentila 90 a timpului de interogare creștea de la câteva secunde la minute pe măsură ce volumul de date creștea.

\section{Graphite și Whisper}

\textbf{Graphite} este o suită de monitorizare compusă din trei componente: Carbon (serviciul de ingestie), Whisper (formatul de stocare) și Graphite Web (interfața de interogare) \cite{graphite}.

\subsection{Formatul Whisper}

Whisper utilizează un format de stocare \textit{Round-Robin Database} (RRD). Caracteristicile principale:

\begin{itemize}
    \item \textbf{Dimensiune fixă}: Fiecare serie temporală ocupă un spațiu fix pe disc, determinat la creare;
    \item \textbf{Intervale fixe}: Timestamp-urile sunt presupuse să vină la intervale exacte;
    \item \textbf{Suprascrierea datelor vechi}: Când spațiul se umple, datele noi le suprascriu pe cele mai vechi.
\end{itemize}

\subsection{Limitări}

\begin{itemize}
    \item \textbf{Nu suportă jitter}: Dacă un punct de date vine la secunda 61 în loc de 60, trebuie fie ignorat, fie interpolat;
    \item \textbf{Stocare pe disc}: Similar cu OpenTSDB, latența este limitată de I/O-ul discului;
    \item \textbf{Fiecare serie într-un fișier separat}: La miliarde de serii, managementul fișierelor devine problematic.
\end{itemize}

\section{Tehnici de compresie pentru serii de timp}

Există multe abordări pentru compresia seriilor de timp, fiecare cu propriile dezavantaje:

\subsection{Compresie cu pierderi (Lossy)}

Tehnicile cu pierderi reduc volumul de date prin aproximare:

\begin{itemize}
    \item \textbf{Downsampling}: Păstrarea doar a unui punct din N (de exemplu, un punct pe minut în loc de unul pe secundă);
    \item \textbf{Agregare}: Înlocuirea mai multor puncte cu media, suma sau alte statistici;
    \item \textbf{Piecewise Linear Approximation}: Aproximarea seriei cu segmente de dreaptă.
\end{itemize}

Aceste tehnici sunt utile pentru date istorice, dar inacceptabile pentru monitorizare în timp real, unde fiecare punct poate indica o problemă critică.

\subsection{Compresie fără pierderi (Lossless)}

Tehnicile fără pierderi păstrează exact datele originale:

\begin{itemize}
    \item \textbf{Delta encoding}: Stocarea diferențelor între valori consecutive în loc de valorile absolute;
    \item \textbf{Run-length encoding (RLE)}: Compresia secvențelor de valori identice;
\end{itemize}

\subsection{Compresie pentru numere în virgulă mobilă}

Compresia valorilor \textit{floating-point} prezintă următoarele provocări:

\begin{itemize}
    \item Reprezentarea IEEE 754 distribuie informația pe 64 de biți (semn, exponent, mantisă);
    \item Diferențele mici între valori pot produce reprezentări binare foarte diferite;
\end{itemize}

Lindstrom și Isenburg \cite{lindstrom2006} au propus o tehnică bazată pe predicție și codare XOR pentru compresia datelor științifice în virgulă mobilă. Gorilla adaptează această tehnică pentru streaming în timp real.

\section{Poziționarea algoritmului Gorilla}

Gorilla ocupă o nișă distinctă în peisajul TSDB-urilor:

\begin{table}[h]
\centering
\begin{tabular}{@{}lccc@{}}
\toprule
\textbf{Caracteristică} & \textbf{OpenTSDB} & \textbf{Graphite} & \textbf{Gorilla} \\
\midrule
Stocare & Disc (HBase) & Disc (Whisper) & Memorie \\
Latență citire & Secunde & Secunde & Milisecunde \\
Compresie & Minimă & Fără & Delta-of-delta + XOR \\
Jitter timestamp & Da & Nu & Da \\
Prioritizare date recente & Nu & Nu & Da \\
Persistență & Da & Da & Opțional (cache) \\
\bottomrule
\end{tabular}
\caption{Comparație între sisteme TSDB}
\label{tab:tsdb-comparison}
\end{table}

Gorilla este gândit ca un \textbf{cache write-through} în fața unui TSDB persistent precum HBase. 

Un write-through cache este un mecanism unde datele sunt scrise simultan în cache și în mediul de stocare permanent (back-end storage).
Când o aplicație trimite un punct de date (un datapoint), acesta este stocat imediat în memoria RAM a lui Gorilla pentru interogări ultra-rapide și, în paralel, este trimis către un sistem persistent (precum HBase) pentru arhivare pe termen lung.
Cache-ul este mereu „la zi” cu baza de date. Dacă interogarea caută date recente, Gorilla le oferă instantaneu fără a mai accesa discul.

Aceste caracteristici au făcut din Gorilla sistemul preferat pentru monitorizare în timp real la Facebook, servind peste 85\% din interogările către datele din ultimele 26 de ore.